\chapter{\uppercase{Background on Data-Consistent Inversion} \label{chapter:02}}

\section{Notation, Terminology, and Assumptions}
\subsection{Models and Parameters}
We begin by assuming that a (deterministic) model, denoted by $$\M (u, \param) = 0,$$ is specified to relate observable state variables $u$ to model inputs ({\em parameters}) denoted by the vector $\param\in\RP$.
The components $\param_\iparam$ may include parameters in either the model operator (e.g. a diffusion coefficient) or input data (e.g. the frequency of a sinusoidal source, initial, or boundary information).
We let $\pspace$ denote the set of all possible input parameters.
We assume $\pspace\subset\RP$ is equipped with a (dominating) measure, $\pmeas$ on the Borel $\sa$ $\pborel$, defining the measure space $\Pspace$.
The solution operator of the model $\M$ then defines a map taking $\param\in\pspace$ to a solution, denoted $u\lam$, to make explicit the dependence on $\param$, which is assumed to be unique.

However, in real experimental settings we are often unable to fully observe $u\lam$. 
Instead, we often only have access to some finite set of observable scalar quantities.
For example, in experiments involving the diffusion of heat, we can typically only record the temperature at some small number of pre-specified points in space-time where measurement devices can be positioned.

\subsection{Quantities of Interest}
Such observable values of $u\lam$ are mathematically modeled by functionals of the solution, denoted $\qoi_\idata: u\lam \to \RR$.
The collection of such functionals into a vector defines a {\em Quantity of Interest} (QoI) map.
Since the solution to the model depends on $\param$, so do the QoI, which motivates the notation
$$\qlam := \qoi( u\lam ) \in\RD,$$
to make this dependence on model parameters explicit.
Furthermore, this convention captures a realistic limitation of an experimental setting, where we may be able to control $\param$ in order to observe $\qlam$, but lack the ability to observe $u\lam$ directly.
The outputs of the QoI map $\qlam = \data$ are what we refer to as the \emph{data}.
Similarly, the range of the QoI defines the \emph{data space} $\dspace$, i.e.
$$\dspace = \qoi(\pspace) \subset \RD$$

We let $\qspace$ denote the set of possible QoI maps for which it is possible to collect experimental data.
For example, suppose we may record only a single temperature measurement at any of ten locations in space-time.
Then $\qspace$ is defined by ten possible QoI maps.
If we can record any two such measurements, then $\qspace$ is defined by $\binom{10}{2} = 45$ possible maps.
Observe that $\qspace$ could easily be uncountable, for example if we were not limited to the spatial locations (or time) at which we could record temperature measurements.
However, for simplicity, we will only discuss problems where $\qspace$ is finite.
In the event that we need to compare maps, we adopt the notation $\dspace_{\qoi}$ to emphasize that the data space depends on the choice of QoI map $\qoi$; when the context is clear, we drop the subscript.
The only assumption on $\qoi$ that we impose throughout this work is that of piecewise-differentiability.

\subsection{Chapter Outline}
First, to respect the order in which these results were researched and developed, we discuss set-based inversion.
In the interest of clarity of exposition, we have adapted the notation of a sample-based approach developed later to express the construction of the former.
Consequently, the references cited may require some work translating. Similarities and differences between the two approaches has heretofore avoided formal documentation, and this work serves to lay the groundwork for such a comparison. 

In this chapter, we discuss the issues that arise from the need to numerically approximate the solutions to our inverse problems but study the implications in \ref{sec:set-error} and \ref{sec:sample-error}.
Discussion of the impact of sample size are kept to a minimum in this chapter in the interest of restricting the scope of chapter \ref{chapter:03} to the implications for convergence of (consistent) solutions to \eqref{eq:inverse-problem}.

%%%%%%%%%%%%%%%%%%%%%%%%%%%%%%%%%%%%%%%%%%%%%%%%%%%%%%%%%%%%%%%%%%%%%%
\pagebreak
\section{Set-Based Inversion for Measures}\label{sec:ch02-set}
% Intro
To properly summarize the Stochastic Inverse Problem (SIP) and desired solution, we define several measure/probability spaces and refer to the schematic given in Figure \ref{fig:scheme} in order to illustrate the steps and spaces required in the formulation and solution of the SIPs we consider herein.
For a more extensive review, we refer the reader to \cite{BBE11}, \cite{BES12}, and \cite{BET+14}.
Additional background and extensions of this theory are available in the PhD theses of Lindley Graham (UT Austin), Scott Walsh (CUD), Lei Yang (CSU) [TK - cite 3].

%%%%%%%%%%%%%%%%%%%%%%%%
\begin{figure}[!h]
\begin{equation}
\underbrace{
\underbrace{
\overbrace{
 \Pspace \xmapsto{\  \qoi \ } \Dspace
  \xmapsto{\ \observedP \ } \Ospace
 }^{
 \text{(S1): Stochastic Inverse Problem (SIP)}
 }
 \xmapsto{\ \qoi^{-1} \ } (\pspace, \cborel, \contourP)
 }_{
 \text{(S2): Solution to SIP Satisfying Eq. \eqref{eq:dataspace_pushforward_measure}}}
 \xmapsto{\ \set{\PP_\ell}_{\ell\in\mathcal{L}} \ } (\pspace, \pborel, \paramP)
 }
 _{
 \text{(S3): Unique Solution to SIP by Eq.~\eqref{eq:disintegration_measure} and Ansatz}
 }
\end{equation}
\caption{The first step (S1) defines (i)~the formulation of the SIP by specification of the model, (ii)~the measure spaces of parameters and (iii)~observable outputs, and (iv)~the probability measure on the latter. The second step (S2) defines a unique solution to the SIP on the space $\pspace$ equipped with the contour $\sa$ $\cborel$ using the definition of the push-forward measure. In (S3), the Disintegration Theorem and and Ansatz are applied to define a unique solution on the space of interest $(\pspace, \pborel)$ equipped with a probability measure $\paramP$.}
\label{fig:scheme}
\end{figure}


The initial measure/probability spaces involved in the formulation of the SIP are summarized in step (S1) of Fig.~\ref{fig:scheme}, starting with measure space $\Pspace$.

The assumption that $\qoi$ is at least piecewise-differentiable implies the measurability of the QoI map, so that the space $\dspace$ induced by $\qoi$ is equipped with the Borel $\sa$ $\dborel$ [TK - cite textbook].
The ``push-forward'' measure $\dmeas$ on ${(\dspace, \dborel)}$ is defined as

\begin{equation}\label{eq:dataspace_pushforward_measure}
\dmeas (A) = \int_A \, d\dmeas := \int_{\qoi^{-1}(A)} \, d\pmeas = \pmeas \left (\qoi^{-1}(A) \right ) \quad \forall \;  A\in\dborel,
\end{equation}

\noindent which defines the measure space $\Dspace$\footnote{When referring to properties of the data space that are not unique to the choice of map used to induce $\dspace$, we will drop the subscript notation and assume the dependence is understood, as expressed in Fig.~\ref{fig:scheme}.}.

In practice, when $\dmeas$ is absolutely continuous with respect to the $\dimD$--dimensional Lebesgue measure, we substitute the Lebesgue measure for $\dmeas$.

The final step in (S1) involves the specification of a probability measure $\dataP$ (absolutely continuous with respect to $\dmeas$) on ${(\dspace, \dborel)}$ to model the uncertainty in data.
This leads to the following SIP: determine a probability measure $\paramP$ on ${(\pspace, \pborel)}$ such that the push-forward measure of $\paramP$ matches $\dataP$.

In other words, determine a $\paramP$ satisfying
\begin{equation}\label{eq:inverse_measure}
\paramP \left ( \qoi^{-1}(E)\right ) = \dataP(E) \; \forall \; E \in \dborel.
\end{equation}

We call any such solution $\paramP$ to Eq.~\eqref{eq:inverse_measure} a (measure-theoretic) solution to the SIP.
This equation implies that any solution is uniquely determined on the induced contour $\sa$
\begin{equation}\label{eq:contour_sa}
\cborel = \set{\qoi^{-1}(E) : E \in \dborel } \subset \pborel,
\end{equation}
which is summarized as step (S2) of Fig.~\ref{fig:scheme}.

However, for sets $A \in \pborel \setminus \cborel$, more information is required than is provided in Eq.~\eqref{eq:inverse_measure} in order to determine $\paramP (A)$.
By the Implicit Function Theorem, if $\qlam \in C^1 (\pspace)$ and we let $\data\in\dspace$ be a fixed datum, $\qoi^{-1}(q)$ exists as a $(\nparams-\ndata)$\--dimensional manifold (possibly piecewise-defined) that we refer to as a \emph{generalized contour} \cite{BET+14}.
These generalized contours can be indexed by a $\dimD$--dimensional manifold (also possibly piecewise-defined) of dimension $\ndata$ called a \emph{transverse parameterization} that intersects each contour once and only once.
In \cite{BET+14}, it is shown that transverse parameterizations are guaranteed to exist and can be approximated by a finite number of $\dimD$---dimensional hyperplanes when $\pspace$ is compact.
In general, the transverse parameterization is not unique. 

We let $\LL$ denote any particular transverse parameterization.
Each $\ell\in\LL$ corresponds to a unique generalized contour $\CC_\ell \in \pspace$ and each point $\param\in\pspace$ belongs to a unique $\CC_\ell\in\pspace$.
Thus, a transverse parameterization defines a bijection between the manifold $\LL$ and the partitioning of $\pspace$ into generalized contours that decomposes $\pspace$ in terms of equivalence classes.
The induced $\sa$ $\cborel$ and this bijection can then be used to define the measurable space $(\LL, \BB_\LL)$.

We denote the projection map $P_\LL : \pspace \to \LL$, and let $\set{\CC_\ell}_{\ell\in\LL}$ represent the family of generalized contours indexed by $\LL$, yielding the associated family of measurable spaces $\set{\left ( \CC_\ell, \BB_{\CC_\ell} \right )}_{\ell\in\LL}{}$.
A Disintegration Theorem [TK - cite] is then leveraged to define a unique decomposition for any $\paramP$ defined on $(\pspace, \pborel)$ as a (marginal) probability measure $\PP_\LL$ on $(\LL, \BB_\LL)$ and a family of (conditional) probability measures $\set{\PP_\ell}_{\ell\in\LL}$ on $\set{\left ( \CC_\ell, \BB_{\CC_\ell} \right )}_{\ell\in\LL}$ such that
\begin{equation}\label{eq:disintegration_measure}
\paramP (A) = \int_{P_\LL(A)} \left ( \int_{P_{\LL}^{-1} (\ell) \cap A}\, d\PP_\ell(\param) \right )\, d\PP_\LL (\ell), \; \forall \; A \in \pborel
\end{equation}

The uniqueness of a probability measure $\paramP$ on ${(\pspace, \cborel)}$ satisfying Eq.~\eqref{eq:inverse_measure} implies the uniqueness of the marginal probability measures $\PP_\LL$ for any particular specification of $\dataP$ on ${(\dspace, \dborel)}$.
The disintegration of Eq.~\eqref{eq:disintegration_measure} implies that a specification of a family of conditional probability measures $\set{P_\ell}_{\ell\in\LL}$ gives us a unique solution to the SIP on ${(\CC_\ell, \BB_{\CC_\ell})}$.

However, the conditional measures cannot be determined solely by the specification of $\dataP$.
We follow the work of \cite{BET+14} and adopt the \emph{standard ansatz} determined by the disintegration of the measure $\pmeas$ to compute probabilities of sets contained within contour events whenever $\pmeas(\pspace) < \infty$, e.g. when $\pmeas$ is the $\dimP$--dimensional Lebesgue measure and $\pspace \in \RP$ is precompact.
The standard ansatz is given by

\begin{equation}\label{eq:standard_ansatz}
\PP_\ell = \mu_{\CC_\ell} / \mu_{\CC_\ell}(\CC_\ell), \; \forall \; \ell \in \LL,
\end{equation}

\noindent where $\mu_{\CC_\ell}$ is the disintegrated volume measure on generalized contour $\CC_\ell$.
Thus, we have defined a unique solution to the SIP on ${(\pspace, \pborel)}$, completing step (S3) in Fig.~\ref{fig:scheme}.

In the absence of other information about differences in relative likelihoods of parameters, the standard ansatz effectively implies a uniform distribution describing the initial state of uncertainty about the input parameters\footnote{In the event that $\pspace$ is compact.}.
In this context, the measure $\paramP$ can be viewed as updating an initial uniform distribution on $\pspace$ in directions informed by the Quantity of Interest map, given uncertain data characterized by $\dataP$.
We will begin to use $\updatedP$ to denote $\paramP$ going forward with this understanding.

\vfill
\subsubsection{Alternative Derivation Using Bayes' Rule}\label{sec:set_bayes}
In the measure-theoretic approach studied in~\cite{BBE11, BET+14}, Voronoi-cell discretizations of $\pspace$ are used to construct set-valued approximations of the updated measure directly, so we refer to it as the \emph{explicit} approach.
By contrast, sampling from densities is an \emph{implicit} approach, and is discussed in greater detail in \ref{sec:ch02-sample}.
Here, we provide a ``set-based'' derivation of the updated measure to more easily compare to the explicit approximation of the solution given measure in~\cite{BET+14}.

First, we start by observing that if $A, B \subset \pspace$ such that $A = \qoi^{-1}(\qoi(B))$, then we have that $B\subset A$ (the inclusion may be proper).
Therefore, for any probability measure $P$ on $(\pspace, \pborel)$, 
\[
P(B) = P(B|A) \, P(A).
\]
If $P$ is intended to solve the inverse problem, then we are motivated to take
\[
P(A) = \observed (\qoi(A)) = \observed (B),
\]
in the above formula.

We must now determine how to properly define $P(B|A)$. 
We leverage Bayes' Theorem~\cite{Smith} in order to utilize the prior density on contour events.
In other words, we use the prior (ansatz) measure $\initialP$ extended on $(\pspace, \pborel)$ and Bayes' Theorem to get
\begin{equation}\label{eq:bayes_full}
P(B|A) = \initialP(B|A) = \frac{ \initialP(A|B) \initialP(B) }{ \initialP(A) },
\end{equation}
and since $B \subset A$, $\initialP(A|B) = 1$, \eqref{eq:bayes_full} simplifies to

\begin{equation}\label{eq:bayes}
\initialP(B|A) = \frac{ \initialP(B) }{ \initialP(A) }.
\end{equation}

Recall from \eqref{eq:predicted} that $\predictedP$ is the push-forward of the initial measure, giving $\initialP(A) = \predictedP (\qoi(A)) = \predictedP \left (\qoi(B)\right )$, which then gives the following set-valued ``solution'' to the stochastic inverse problem:
\begin{equation}\label{eq:sip_sol_cont}
\updatedP(B) := \begin{cases}
\initialP(B) \frac{ \observedP(B) }{ \predictedP \left (\qoi(B)\right ) ) } & \text{ if } \initialP(B) > 0,\\
0 & \text{ otherwise}.
\end{cases}
\end{equation}

This set-valued update is only a solution on certain (sub-)$\sigma$-algebras of $\pborel$ for which $B$ must belong to apply \eqref{eq:sip_sol_cont}. 
Nonetheless, we can form explicit approximations to this measure, e.g. as done in~\cite{BET+14, BES12, BBE11}. 
In other words, an Ansatz is used in place of the prior; it serves the same purpose to distribute probabilities in directions not informed by the QoI map.

% However, such an explicit approach requires an approximation of \emph{events in $\pborel$}. 
% This is in direct contrast to the numerical approximation of the density $\updated$ that only requires approximation of $\predicted$.
% Since we often expect the dimension of $\dspace$ to be less than the dimension of $\pspace$, this can prove to be a significant numerical advantage for the ``implicit'' approximation given by the updated probability density function. 

\vfill
% Numerical Approximation
\subsection{Numerical Approximation and Analysis}\label{sec:set-algorithm}
We present a non-intrusive algorithm based on Monte-Carlo sampling\---initially introduced in \cite{BET+14} and further analyzed in \cite{BET+14-arxiv}\---that is structured in four stages (written as four independent for-loops) that are linked to the stages in Fig.~\ref{fig:scheme}.
We direct the interested reader to \cite{BET+14-arxiv} for more detailed information and analysis of this algorithm, e.g., on the requirement of a sampler being ``$\pborel$-consistent'' to ensure convergence.


\begin{algorithm}[hbtp]
\DontPrintSemicolon
Choose a discretization partition $\set{D_\idisc}_{\idisc=1}^{\ndiscs}$ of $\dspace$.\\
	\For{$\idisc = 1, \hdots, \ndiscs$}{
			Compute $p_{\dspace, \idisc} = \dataP(D_\idisc)$.
	}
	Choose samples $\set{\param^{(\iparam)}}_{\iparam=1}^{\nsamps} \subset \pspace$, which implicitly defines a Voronoi-cell partition $\set{\VV_\iparam}_{\iparam=1}^{\nsamps}$ of $\pspace$.\\
	\For{$\iparam = 1, \hdots, \nsamps$}{
	Compute $\qoi_\iparam = \qoi(\param^{(\iparam)})$.\\
	Let $\OO_\idisc = \set{\iparam: \qoi_\iparam \in D_\idisc}$.\\
	Compute approximations $V_\iparam \approx \pmeas (\VV_\iparam)$.
	}
	\For{$\idisc = 1, \hdots, \ndiscs$}{
	Compute $\CC_\idisc = \set{\iparam:Q_\iparam \in D_\idisc}$.
	}
	\For{$\iparam = 1, \hdots, \nsamps$}{
	Compute $p_{\pspace, \iparam} = \left ( V_\iparam / \sum_{j\in \CC_{\OO_\iparam} } V_j \right ) p_{\dspace, \OO_\iparam}$.
	}
	For any $A\in \pborel$, compute
	\begin{equation}
	\PP_{\pspace, \ndiscs, \nsamps} (A) = \sum_{\iparam=1}^\nsamps p_{\pspace, \iparam} \Chi_{\VV_\iparam} (A)
	\end{equation}
 \caption{Numerical Approximation of the Inverse Density}
 \label{alg:inv_density}
\end{algorithm}
\FloatBarrier

The first two stages correspond to formulating the discretized version of the SIP given in step (S1) in Fig.~\ref{fig:scheme}.
We first discretize the probability space $\Ospace$.
Then, we simultaneously discretize the measure space $(\pspace, \pborel, \pmeas)$ and construct a simple-function approximation to the map $\qoi$.
These stages introduce the primary sources of error, and the third and fourth stages may be thought of as solving the discretized SIP exactly.
The samples that are used to describe $\pspace$ implicitly define a set of Voronoi cells $\set{\VV_\iparam}_{\iparam=1}^{\nsamps}$, which can be seen in Figure~\ref{fig:voronoi_cells}.
Each sample set defines a fundamentally different geometry.

\begin{figure}[ht]
\centering
	\begin{minipage}{.475\textwidth}
		\includegraphics[width=\linewidth]{./images/voronoi_diagrams/voronoi_diagram_N25_r0}
	\end{minipage}
\caption{
Voronoi-cell discretization (partition) induced by $\nsamps = 25 $ uniform i.i.d.~random samples in $\pspace = [0,1]^2$.
}
\label{fig:voronoi_cells}
\end{figure}

The third stage then identifies the collection of Voronoi cells in $\pspace$ that approximate the contour events in $\cborel$ defined by $\qoi^{-1}(D_\idisc)$ for $\idisc=1,\hdots,\ndiscs$. This allows us to formulate the consistent solution to the discretized SIP on $(\pspace, \cborel, \contourP)$ as illustrated in step (S2) of Fig.~\ref{fig:scheme}.
Finally, the fourth stage\---associated with step (S3) in Fig.~\ref{fig:scheme}\---uses a discrete version of the ansatz to approximate the probability of $\VV_\iparam$ for $\iparam=1,\dots,\nsamps$.
This results in an approximate probability measure, denoted by $\PP_{\pspace, \ndiscs, \nsamps}$, which produces the same probability estimates for events $A$ and $A\setminus \set{ \param^{(\iparam)} }_{\iparam=1}^\nsamps$, which are identical almost everywhere with respect to $\pmeas$.

Note that Algorithm~\ref{alg:inv_density} makes no mention of the method by which the samples $\set{ \param^{(\iparam)} }_{\iparam=1}^{\nsamps}$ were generated or sets in $\set{D_\idisc}_{\idisc=1}^{\ndiscs}$ are chosen.
$\set{ \param^{(\iparam)} }_{\iparam=1}^{\nsamps}$ may be generated using uniform random sampling, latin hypercube sampling, or even regular grids.
A thorough discussion of the choices involved in making such decisions is beyond the scope of this work, though we touch briefly on the discretization of $\dspace$ below.

\vfill
% Descriptions of Error
\subsection{Descriptions of Error}\label{sec:set-error}
Recall that we assumed $\dataP$ is absolutely continuous with respect to $\dmeas$, which allows us to describe $\dataP$ with a density $\rho_\dspace$. Then, for any partition $\set{D_\idisc}_{\idisc=1}^{\ndiscs}$ of $\dspace$,
\[
\dataP (D_\idisc) = \int_{D_\idisc} \rho_\dspace \, \dmeas, \quad \text{ for } \idisc = 1, \hdots, \ndiscs.
\]

We often use Monte Carlo approximations to compute the approximations $p_{\dspace, \idisc}=\dataP(D_\idisc)$ in the first for-loop in Algorithm~\ref{alg:inv_density}.
These samples are generated on $\dspace$ and do not require numerical solutions to the model.
We therefore assume that for any discretization of $\dspace$, these approximations can be made sufficiently accurate and neglect the error in this computation.

We denote the exact solution to the SIP associated with this partitioning of $\dspace$ by $\PP_{\pspace, \ndiscs}$.
In situations where $\qoi(\param^{(\iparam)})$ is estimated (e.g. by application of a functional on a finite-element solution to a PDE), the approximate solutions to the SIP given in the final for-loop of Algorithm~\ref{alg:inv_density} are denoted by $\PP_{\pspace, \ndiscs, \nsamps, h}$.
Here, the $h$ is in reference to a mesh or other numerical parameter that determines the accuracy of the numerical solution $u_h(\param^{(\iparam)})\approx u(\param^{(\iparam)})$, and subsequently the accuracy in the computations of $\qoi_\iparam = \qoi(\param^{(\iparam)})$ in Algorithm~\ref{alg:inv_density}.

We assume that $h$ is tunable so that for any $A\in \pborel$, 
\[
\lim\limits_{h \downarrow 0} \PP_{\pspace, \ndiscs, \nsamps, \imesh} (A) = \PP_{\pspace, \ndiscs, \nsamps} (A).
\]
It is possible to prove the convergence of $\PP_{\pspace, \ndiscs, \nsamps, \imesh} (A) \to \paramP (A)$ for some $A\in \pborel$ and on estimating the error in $\PP_{\pspace, \ndiscs, \nsamps, h}(A)$.
For example, in \cite{BGE+15}, adjoint-based a posteriori estimates in the computed QoI are combined with a statistical analysis to both estimate and bound the error in $\PP_{\pspace, \ndiscs, \nsamps, \imesh} (A)$.
In [TK - cite ISNME 2019], adjoints are used to compute both error and derivative estimates of $\qoi(\param^{(\iparam)})$ to improve the accuracy in $\PP_{\pspace, \ndiscs, \nsamps, \imesh} (A)$.
However, no work has to date fully explored the \emph{convergence rates} of Algorithm \ref{alg:inv_density}.
Furthermore, no work has yet to establish that these rates are independent of the choice of QoI map despite other studies establishing that the absolute error is very much affected by the geometric properties of the QoI maps [TK - cite Lindley + Butler].

In order to study convergence, we need to define a notion of distance on the space of probability measures on $\pspace$, which we denote by $\PPspace$.
% There are many choices available to us and we discuss several useful metrics on $\paramP$ in Section~\ref{sec:metrics}.
We use the Total Variation metric (TV) throughout this work, but for the time being, let $d$ represent any metric on $\PPspace$.

Then, by repeated application of the triangle inequality,
\begin{equation}
\label{eq:set-triangleineq}
d(\PP_{\pspace, \ndiscs, \nsamps, h}, \paramP) \leq
\underset{ \text{(E1)} }{\underbrace{d(\PP_{\pspace, \ndiscs, \nsamps, h},\PP_{\pspace, \ndiscs, \nsamps})}} +
\underset{ \text{(E2)} }{\underbrace{d(\PP_{\pspace, \ndiscs, \nsamps}, \PP_{\pspace, \ndiscs}) }}+
\underset{ \text{(E3)} }{\underbrace{d(\PP_{\pspace, \ndiscs}, \paramP) }}.
\end{equation}

The term (E1) describes the effect of the error in the numerically evaluated $\qoi_\iparam$ on the solution to the SIP.
The term (E2) describes the effect of finite sampling error in $\pspace$ on the solution to the SIP and (E3) describes the effect of discretization error of $\dataP$ on the solution to the SIP.

\vfill
% Example
\subsection{Example}\label{sec:set-example}

TK - Introduce, 2D identity map, known observed (square in center with area 1/100, i.e. density value of 100).

\begin{python}
"""
Set up and solve problem with identity map
"""
# import libraries
import bet.sample as sample
import bet.sampling.basicSampling as bsam
import bet.calculateP.simpleFunP as simpleFunP
import bet.calculateP.calculateP as calculateP
import numpy as np
import scipy.stats as sstats
# define input space parameters and model to instantiate sampler object
dimension = 2
numSamples = 100
I = np.eye(dimension)
def model(input_samples):
        return (I@input_samples.T).T
sampler = bsam.sampler(model)
# instantiate objects that hold input/output samples
# default random sample set is uniform over unit domain (normalized space)
input_set = input_samples = bsam.random_sample_set('r',input_obj=dimension, num_samples=numSamples)
param_ref = np.array([0.5, 0.5])
input_set.set_reference_value(param_ref)
# Estimate volumes of Voronoi cells associated with the parameter samples
if MC_assumption is False:
    input_samples.estimate_volume(n_mc_points=5E4)
else:
    input_samples.estimate_volume_mc()

# input_set = bsam.regular_sample_set(input_obj=dimension, num_samples_per_dim=49)
disc = sampler.compute_QoI_and_create_discretization(input_sample_set=input_set)
Qref = disc.get_output().get_reference_value()
print('Reference Value:', param_ref, 'maps to', Qref)
# define inverse problem
disc_1 = disc.copy()
simpleFunP.regular_partition_uniform_distribution_rectangle_size(
        data_set=disc_1, Q_ref=Qref, rect_size=0.2,
        cells_per_dimension = 1)
calculateP.prob(disc_1)
# compare with higher-fidelity discretization of output space
disc_2 = disc.copy()
simpleFunP.regular_partition_uniform_distribution_rectangle_size(
        data_set=disc_2, Q_ref=Qref, rect_size=0.2,
        cells_per_dimension = 2)
calculateP.prob(disc_2)
\end{python}

Note that there is no need to explictly call {\tt disc.compute\_pushforward()}, (or \pythoninline{disc.compute_predicted()}) since it is computed automatically if none have been previously constructed.
When \pythoninline{disc.updated_pdf()} is called, densities are evaluated at the initial set of $\nsamps$ random samples, and stored in \pythoninline{disc._input_sample_set._densities}.
However, the function \pythoninline{disc.predicted_pdf()} is capable of evaluating the solution at any new set of samples (provided a model is available/equipped to the discretization), something we leverage for plotting on a regular grid.

Once our four discretization objects \pythoninline{disc}, \pythoninline{disc_a}, \pythoninline{disc_b} and \pythoninline{disc_c} have been generated, we can use some utility plotting functions to compare the densities:

\begin{python}
"""
Plotting code to generate figures.
"""
# define plotting parameters
nbins = 50
xmn, xmx = 0.25, 0.75
ymn, ymx = 0.25, 0.75
xi, yi = np.mgrid[xmn:xmx:nbins*1j, ymn:ymx:nbins*1j]
# plotting functions computes nearest-neighbors to
# the regular grid of samples.
plot_2d_comparison(xi, yi, disc_1, disc_2,
                   '$M=1, N=%d$'%(numSamples),
                   '$M=4, N=%d$'%(numSamples))
\end{python}

\begin{figure}[ht]
\begin{minipage}{.975\textwidth}
  \includegraphics[width=\linewidth]{./examples/identity/set/M1-N100_N100-vs-M4-N100_N100.pdf}
\end{minipage}
\caption{
$\nsamps=100$ were used to discretize $\pspace$ and $\ndiscs=1, 4$ (left/right) were used to discretize $\dspace$.
The latter was chosen to visualize the geometry of describing $\pspace$ with $\pborel$.
}
\label{fig:ex:identity_set_1E2}
\end{figure}

We have chosen a uniform density to describe the uncertainty in our output space.
Any value that is within $0.1$ to the left and right of the reference value \pythoninline{Qref = [0.5, 0.5]} in each dimension is treated as equally likely.
This was done so that using $\ndiscs = 1$ samples to discretize $\dspace$ would fully characterize the characteristic function density representing this uncertainty.
The inverse image of this set is a characteristic function defined on $\pborel$, so errors will exist in particular at the boundaries of the region.

The fundamental challenge with the set-based approach is linked to the geometry of the induced Voronoi-cell tesselation on $\pspace$.
With only $\nsamps = 100$ samples, we see in \ref{fig:ex:identity_set_1E2} that the region (which is supposed to be a square) hardly represents one.
There is ample variation in the shape parameters of the induced sets $\VV_\iparam$ when so few samples are used.
It is possible to ``get lucky'' with the aligning of boundaries between the true target density $\Chi_{[0.4, 0.6]^2}$, but the Figure is representative of the difficulty of using $\nsamps = 100$ random samples to describe a geometry.
With $\nsamps = 1000$, there are usually still significant differences in the symmetric difference of approximated and true supports of the densities.

Below, we demonstrate the use of more samples to resolve the geometry of the desired set.
\begin{figure}[ht]
\begin{minipage}{.975\textwidth}
  \includegraphics[width=\linewidth]{./examples/identity/set/M1-N1000_N1000-vs-M4-N1000_N1000.pdf}
  \includegraphics[width=\linewidth]{./examples/identity/set/M1-N10000_N10000-vs-M4-N10000_N10000.pdf}
\end{minipage}
\caption{
(Top):$\nsamps=1,000$ were used to discretize $\pspace$ and $\ndiscs=1, 4$ (left/right) were used to discretize $\dspace$.
(Bottom): The same, except with $\nsamps=10,000$, where we finally begin to see something resembling the correct correct geometry.
}
\label{fig:ex:identity_set_1E3_1E4}
\end{figure}
\FloatBarrier

We remark on the fact that in this particular situation, $\ndiscs = 1$ is a ``correct'' choice for the probability measure chosen, and $\ndiscs = 4$ actually introduces errors.
The reason that the latter solutions have different values inside of the support of $\PPspace$


%%%%%%%%%%%%%%%%%%%%%%%%%%%%%%%%%%%%%%%%%%%%%%%%%%%%%%%%%%%%%%%%%%%%%%
\pagebreak
\section{Sample-Based Inversion for Measures}\label{sec:ch02-sample}

% Intro
TK - words here. 
% Numerical Approximation
\subsection{Numerical Approximation and Analysis}\label{sec:sample-algorithm}

% Descriptions of Error
\subsection{Descriptions of Error}\label{sec:sample-error}

KDE is now the primary source, show relevant triangle inequality here.
Summarize Troy and Tim's work on the sensitivity analysis of everything.


Then, we have by repeated application of the triangle inequality that
\begin{equation}
\label{eq:sample-triangleineq}
d(\PP_{\pspace, \ndiscs, \nsamps, h}, \paramP) \leq
\underset{ \text{(E1)} }{\underbrace{d(\PP_{\pspace, \ndiscs, \nsamps, h},\PP_{\pspace, \ndiscs, \nsamps})}} +
\underset{ \text{(E2)} }{\underbrace{d(\PP_{\pspace, \ndiscs, \nsamps}, \PP_{\pspace, \ndiscs}) }}+
\underset{ \text{(E3)} }{\underbrace{d(\PP_{\pspace, \ndiscs}, \paramP) }}.
\end{equation}

Talk more about it.

\vspace{2in}

% Example
Identity Map in two dimensions
\subsection{Example}\label{sec:sample-example}

Talk about the problem setup, code blocks shown.

\begin{python}
"""
Set up and solve problem with identity map
"""
# import libraries
import bet.sample as sample
import bet.sampling.basicSampling as bsam
import numpy as np
import scipy.stats as sstats

# define input space parameters and model to instantiate sampler object
dimension = 2
numSamples = 100
I = np.eye(dimension)
def model(input_samples):
        return (I@input_samples.T).T
sampler = bsam.sampler(model)

# instantiate objects that hold input/output samples
input_set = bsam.random_sample_set('r', dimension, num_samples=numSamples)
disc = sampler.compute_QoI_and_create_discretization(input_set)

# define inverse problem
disc.set_initial(dist=sstats.uniform, loc=0, scale=1, gen=False)
disc.set_observed(dist=sstats.uniform, loc=0.4, scale=0.2)

disc_a = disc.copy()
# set analytical density (dimension is inferred)
disc_a.set_predicted(dist=sstats.uniform, loc=0, scale=1)

# create different sample sets with higher fidelity
disc_b = disc.copy()
disc_b.set_initial(num=int(1E3), gen=True)
# the line above removes the previous predicted distribution,
#     stored densities, probabilities, and volumes.
disc_c = disc.copy()
disc_c.set_initial(num=int(1E4), gen=True)
\end{python}

Note that there is no need to explictly call {\tt disc.compute\_pushforward()}, (or \pythoninline{disc.compute_predicted()}) since it is computed automatically if none have been previously constructed.
When \pythoninline{disc.updated_pdf()} is called, densities are evaluated at the initial set of $\nsamps$ random samples, and stored in \pythoninline{disc._input_sample_set._densities}.
However, the function \pythoninline{disc.predicted_pdf()} is capable of evaluating the solution at any new set of samples (provided a model is available/equipped to the discretization), something we leverage for plotting on a regular grid.

Once our four discretization objects \pythoninline{disc}, \pythoninline{disc_a}, \pythoninline{disc_b} and \pythoninline{disc_c} have been generated, we can use some utility plotting functions to compare the densities:

\begin{python}
"""
Plotting code to generate figures.
"""
# define plotting parameters
nbins = 50
xmn, xmx = 0.25, 0.75
ymn, ymx = 0.25, 0.75
xi, yi = np.mgrid[xmn:xmx:nbins*1j, ymn:ymx:nbins*1j]
# plotting functions call .get_updated(), which re-computes
# the pushforward distribution on a regular grid of samples
plot_2d_comparison(xi, yi, disc, disc_a, '$N=$100', 'Analytical')
plot_2d_comparison(xi, yi, disc_b, disc_c, '$N=$1,000', '$N=$10,000')
for d in [disc, disc_b, disc_c]:
    udpated_pdf_conditional_comparison(d, num=100, condition_on=0.5, label='approx')
\end{python}

\begin{figure}[ht]
\begin{minipage}{.975\textwidth}
\includegraphics[width=\linewidth]{./examples/identity/samp/N100_N100-vs-Analytical_N100.pdf}
\end{minipage}
\caption{
(Left): $\nsamps=100$ were used to construct the predicted distribution $\predicted$.
(Right): By specifying an analytical $\predicted$, the effect of using $\nsamps$ to approximate a pushforward distribution disappears. The problem can be fully specified in BET without any random sampling.
}
\label{fig:ex:identity_sampling_exact}
\end{figure}

\begin{figure}[ht]
\begin{minipage}{.975\textwidth}
\includegraphics[width=\linewidth]{./examples/identity/samp/N1-000_N1000-vs-N10-000_N10000.pdf}
\end{minipage}
\caption{
$\nsamps=1,000$ (left) and $\nsamps=10,000$(right) were used to construct the predicted distribution $\predicted$.
There is no signficant error in estimating the support of the distribution, only the density approximation itself.
}
\label{fig:ex:identity_sampling_approx}
\end{figure}

\begin{figure}[ht]
\centering
% N = 1E2
\begin{minipage}{.975\textwidth}
		\includegraphics[width=0.5\linewidth]{./examples/identity/samp/identity_1d_conditionals_50E-2_N100_approx.pdf}
    \includegraphics[width=0.5\linewidth]{./examples/identity/samp/identity_1d_conditionals_50E-2_N100_approx.pdf}
\end{minipage}
% N = 1E3 
\begin{minipage}{.975\textwidth}
		\includegraphics[width=0.5\linewidth]{./examples/identity/samp/identity_1d_conditionals_50E-2_N1000_approx.pdf}
		\includegraphics[width=0.5\linewidth]{./examples/identity/samp/identity_1d_conditionals_50E-2_N1000_approx.pdf}
\end{minipage}
% N = 1E4 
\begin{minipage}{.975\textwidth}
		\includegraphics[width=0.5\linewidth]{./examples/identity/samp/identity_1d_conditionals_50E-2_N10000_approx.pdf}
		\includegraphics[width=0.5\linewidth]{./examples/identity/samp/identity_1d_conditionals_50E-2_N10000_approx.pdf}
\end{minipage}
\caption{
(Left): $\param_1=0.5$ conditional.
(Right): $\param_2=0.5$ conditional.
(Top to Bottom): Conditional for $\updated$ solutions for  $\nsamps=1E2, 1E3, \text{and} 1E4$ random samples.
}
\label{fig:identity_sampling_conditionals}
\end{figure}
\FloatBarrier


[TK - Words]

\begin{equation}
\updatedP = \initialP \frac{\observedP}{\predictedP}
\end{equation}

\begin{equation}
\begin{split}
\dci\\
\dciP\\
\dciD
\end{split}
\end{equation}

In place of the ansatz, we have an initial distribution.


%%%%%%%%%%%%%%%%%%%%%%%%%%%%%%%%%%%%%%%%%%%%%%%%%%%%%%%%%%%%%%%%%%%%%%

\section{Software Contributions}\label{sec:ch02-software}
\subsection{Background and Motivation}
The open-source software package BET was developed actively from 2012-2015 as part of research performed under grant [TK grant-NSF+DOE].
It was originally written in Python 2.7 and is administered by the Computational Hydrology Group at the University of Texas: Austin through their UT-CHG GitHub group [TK - cite Github].
The initial purpose of this open-source software package was to implement the methods first described in [TK - cite BET papers] for the description and solution of stochastic inverse problems summarized in Section~\ref{sec:ch02-set}.

In the intermittent years since its original publication in [TK - date of first release, cite Github], the BET package has seen two major releases and the incorporation of several sub-modules (e.g. the functions in {\tt sensitivity} implement much of the original research performed by Dr. Walsh [TK - cite Scott]).


\subsection{Upgrades, Updates, and Features}
Since the last major release [TK - cite latest release], the Python community announced the end of long-term support for Python 2 [TK - cite announcement].
Several of the dependencies in BET have been actively developed in Python 3 with no updates to the Python 2 analogs, which suggested that BET should likely undergo the same transition.

The work summarized in Section~\ref{sec:ch02-sample} was implemented in Python 3 independently by the author through the release of the ConsistentBayes package.
Since that code was used for many of the preliminary results for this work, it made very little sense to re-implement them in Python 2 for BET given the recent trends in community development.
With funding made available through the NSF [TK - cite grant], the opportunity to upgrade BET to Python 3 was the most sensible choice.


\subsubsection{Version 2.1.0}
The upgrade to Python 3.4+ began in January 2019 as a first step to incorporate the new sample-based method into BET.
It was completed in late February.
Major (minor? version? TK) release 2.1.0 [TK - put in release] was designed to provide backwards-compatibility with the Python 2.7 version.
Future installations (starting in 2020) will not limit the versions of some core dependencies in order to accommodate backwards-compatibility with Python 2 (e.g. {\tt numpy}, {\tt scipy}), since this would likely downgrade previously installed software for end-users.


\subsubsection{Version 2.2.0}
Several releases of BET (after the upgrade to Python 3 in v2.1.0), incorporated developments that will be discussed herein.
For


\subsubsection{Version 2.2.1}
Major bugfix for parallel testing allowed tests to pass for more than 2 processors.
For some tests, this involved changing the setup parameters to ensure the problem was large enough to break up onto up to 8 processors.
For others, siginficant changes had to be made to structure to allow for proper saving and loading of files in parallel.


\subsubsection{Version 2.3.0}
This release incorporated the sampling-based approach discussed in Section~\ref{sec:ch02-sample}.


\subsection{Examples in BET}
Basic plotting functionality of BET is demonstrated in iPython notebooks [TK - some kind of citation here], which have seen an exponential growth rate on GitHub, and can be edited by the end-user to work with different plotting library versions and backends.
These notebooks were originally created to reflect the example suite in BET (which were {\tt .py} files), but later they were migrated into a separate repository BET-examples to allow for better organization.
In the new framework, each notebook functions as an independent example.
Several of these notebooks were adapted from the example code in this thesis repository.




\section{Illustrative Examples}\label{sec:ch02-examples}
In some examples, an analytic, closed-form expression for the QoI map is used.
In these examples, term (E1) in Eq.~\eqref{eq:set-triangleineq} is identically zero.
Furthermore, since the probabilities we introduce on $\dspace$ in the numerical results are uniform and our maps linear, the densities can be described analytically with a characteristic function.
In this event, the solution $\paramP$ to the SIP is given exactly by a change of variables formula and it's support can be specified exactly.
By inverting characteristic functions, the solutions are also be members of this same family of functions if the choice of \emph{ansatz} (or \emph{initial density}) is taken to be uniform over $\pspace$.

Such simplifications in the examples considered here allow us to study the DCI method for a class of functions for which a solution is readily available, and serves as a requisite testing ground before advancing to more nuanced problem definitions.
In a sense, these are both ``unit'' (and ``regression'') tests for various aspects of the computational algorithms (and entire algorithms, respectively).

We present a brief overview of the factors that influence our practical ability to accurately approximate $\paramP$ and $\updated$ using finite sampling.

For the set-based approach discussed in \ref{sec:ch02-set}, it is desirable that $\ndiscs$ is chosen without respect to $\nsamps$ so that (E3) = $d(\PP_{\pspace, \ndiscs}, \paramP)$ from \ref{sec:set-error} is sufficiently small or eliminated entirely.
This amounts to saying that the decision about how to discretize the uncertainty in $\dspace$ is made a priori to cater to some problem specifications.
We choose to impose uniform distributions on $\dspace$ so that the set-valued analog to $\observed$ is perfectly described with $\ndiscs=1$.
This allows for a better comparison between the set- and sample-based approaches by eliminating this variable. 
Therefore, we focus our attention on the source of error introduced by the primary contribution of error in Eq.~\eqref{eq:set-triangleineq}:  discretizing the parameter space, which is represented by the term (E2) = $d(\PP_{\pspace, \ndiscs, \nsamps}, \PP_{\pspace, \ndiscs})$.
The number of samples fundamentally limits the characterization of events due to the resulting simple-function approximation of the density $d\paramP / d\pmeas$, which we compare to $\updated$ from the sample-based approach.

Since there is no error introduced from discretizing $\pspace$ in the sample-based approach from \ref{sec:ch02-sample}, the primary contribution of error in this approach comes from approximating the push-forward density in situations where an analytical $\predicted$ is not known.

\subsection{Exponential Decay}\label{ex:decay-set-sample}

set-based and sample-based, two data points but show two possible choices.
invert the correct set but do make note that in order to do so, we needed to have observed a perfect signal, which is unrealistic.

show a couple different time values.


\subsection{1D Heat Rod}\label{ex:heat-set-sample}
%Here we present a nonlinear example to demonstrate that the general trend of the previous results also holds.
%Some nuanced differences do arise, however, and we address them after the problem statement.
Consider the one-dimensional heat equation with homogeneous Neumann boundary conditions on the unit interval:

\begin{equation}
\begin{split}
\rho c \frac{\partial T}{\partial t} = \nabla \cdot ( \kappa \nabla T) + f(x), \quad & x\in (0,1), t\in (0,1) \\
f(x) = A e^\frac{- (x-0.5)^2}{w} \Chi_{[0,0.5]}(t)
\end{split}
\end{equation}
\emph{Alternative setup: }

\begin{equation}
\begin{cases}
\rho c \frac{\partial T}{\partial t} = \nabla \cdot ( \kappa \nabla T) + f(x,t), & \text{if } x\in \Omega \\
\frac{\partial T}{\partial \vec{n}} = 0 & \text{if } x\in \partial \Omega
\end{cases}
\end{equation}
where $\Omega = (0,1)\times (0,1)$ is the space-time interior and $f(x,t) = A e^\frac{- (x-0.5)^2}{w} \Chi_{[0,0.5]}(t)$.

Here, we interpret the following problem as heating the middle of an infinitesimally thin unit-length rod for half a second with the heat-source modeled by a Gaussian curve with amplitude $A=50$ and variance of $w=0.05$.
The rod is subdivided in two, and each half has an uncertain thermal diffusivity $\kappa \in [0.01, 0.2]$.
This yields a two-dimensional parameter space $\param = (\param_1, \param_2) \in [0.01, 0.2]\times [0.01, 0.2]$, where $\param_1$ represents the thermal diffusion on the left-half and $\param_2$ is the $\kappa$ for the right half.

The quantities of interest we study are four point-evaluations of the state variable, at spatial location 0.25, 0.51, 0.67, and 0.98 along the rod.
Choosing any pair of them for the inversion yields six possible quantities of interest maps.
As before, we demonstrate that some choices appear to have advantages over others.

From the prior examples, we would suspect that choosing the QoI map with lower skewness results in lower Hellinger distances.
However in the earlier experiments we utilized maps that inverted into sets of identical size, which is not the case in this nonlinear example; each QoI map scales sets differently depending on the location in the parameter space.
To isolate this scaling effect, we attempt to compare QoIs that invert into sets of similar size \emph{on average} but have differing average skewness.

This is what motivated our specific choice of spatial locations at which to measure the state variable $T$.
Our first QoI $\qoiA$ uses measurements at 0.25 and 0.51, and has average skewness of 1.08, and our second $\qoiB$ uses measurements at 0.67 and 0.98, with average skewness 1.56.
While we would have liked to use a map with average skewness of 2 for a more similar comparison to the prior examples, this was the best range we could find where the maps inverted into sets of comparable size on average\footnote{average local scaling is 1.99 versus 2.19}.

Owing to the nonlinearity of the problem, the Hellinger distances between reference and estimated probability measures now have an inherent dependence on the location of the point $\param$ in the parameter space.
We ran the simulations for a regular $3\times3$ grid exploring the interior of the parameter space and present a selection of two reference points that illustrate the differences in the nonlinear case.

In the two-dimensional data spaces $\dspaceA$ and $\dspaceB$, our uncertainty is a uniform box centered at $\qoiA(\paramref)$ with side-lengths of 0.1.
When $\paramref$ is the bottom-left corner of our $3\times3$ grid, the two maps produce very different results, with $\qoiA$ outperforming $\qoiB$ in a similar manner as we saw in the linear examples (see Fig.~\ref{fig:NLbotleft}).
When $\param_{\text{ref}}$ is in the upper-center of the grid, the inverse images are similar, as shown in Fig.~\ref{fig:NLtopmid}, and so which map to use for inversion into this part of the parameter spaces is not a clear choice. We might even be tempted to use the more-skewed (on average) map since it inverts into a set with smaller support.

%
%\begin{figure}[h]
%\begin{minipage}{.475\textwidth}
%\includegraphics[width=\linewidth]{./images/pt0Plot-reg_BigN_40000_reg_M_1_rand_I_100000}
%\end{minipage}
%\begin{minipage}{.475\textwidth}
%\includegraphics[width=\linewidth]{./images/pt5Plot-reg_BigN_40000_reg_M_1_rand_I_100000}
%\end{minipage}
%\caption{Comparison of the differences in Hellinger distances for the two maps and two reference points. The results for the bottom-left reference value is shown on the left and the top-center is shown on the right.}
%\label{fig:NLHD}
%\end{figure}
%
%The Hellinger distance plots for these two reference values are compared in Fig.~\ref{fig:NLHD}.
%Of the nine reference $\param$'s we studied, $\qoiA$ yielded no considerable advantage in terms of the number of samples required to approximate the inverse images in three cases (the plots were similar to that in the right of Fig.~\ref{fig:NLHD}).
%In three cases, $\qoiA$ performed just a bit better than $\qoiB$, (somewhere between the two figures in Fig.~\ref{fig:NLHD}).
%In two cases, $\qoiA$ performed better than  $\qoiB$, as in the left of Fig.~\ref{fig:NLHD}.
%In one case (with $\param$ in the bottom right corner), the difference was even more dramatic ($\qoiA$ yielded similar Hellinger distances with less than a fourth the samples).
%
\begin{figure}[h]
\begin{minipage}{.475\textwidth}
\includegraphics[width=\linewidth]{./images/refheat_pt0Q1_M1N40000_2D_0_1}
\end{minipage}
\begin{minipage}{.475\textwidth}
\includegraphics[width=\linewidth]{./images/refheat_pt0Q2_M1N40000_2D_0_1}
\end{minipage}
\caption{The inverse image of the reference measure for $\qoiA$ (left) and $\qoiB$ (right). The latter is visually quite a bit more skewed }
\label{fig:NLbotleft}
\end{figure}

\begin{figure}[h]
\begin{minipage}{.475\textwidth}
\includegraphics[width=\linewidth]{./images/refheat_pt5Q1_M1N40000_2D_0_1}
\end{minipage}
\begin{minipage}{.475\textwidth}
\includegraphics[width=\linewidth]{./images/refheat_pt5Q2_M1N40000_2D_0_1}
\end{minipage}
\caption{The inverse image of the reference measure for $\qoiA$ (left) and $\qoiB$ (right). Here, the local skewnesses are similar, so we do not expect to see much of a difference in the Hellinger distances.}
\label{fig:NLtopmid}
\end{figure}

With these nonlinear cases, we find that taking an ``on average'' approach is inefficient, as there can be dramatic differences in the geometric properties of the inverse images in the parameter space depending on the location.
These results motivate further study into utilizing different QoI maps (perhaps some of those other four combinations available to us in this example) depending on where the samples came from in the parameter space.
In general, we saw in this example that given that two maps invert into sets of similar size on average, using the one with lower skewness results in less samples required to accurately approximate the inverse image.
The maps we used had average skewnesses that differed by 0.5 (instead of by 1), and the trend from the linear examples still held in significant portions of the parameter space.

