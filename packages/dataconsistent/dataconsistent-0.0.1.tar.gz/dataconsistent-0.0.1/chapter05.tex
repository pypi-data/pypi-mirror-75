\chapter{\uppercase{Research Directions} \label{chapter:05}}

Summary of NSF project, some OED stuff perhaps, functional assimilation.

\section{Addressing Model Assumptions}\label{sec:ch05-variance}

What if we don't know the variance? How does mis-estimating it affect our solutions? 

Multiplicative noise - handled in a straightforward way, maybe put an example here and leave it at that? Put it in appendix? 


\section{Leveraging Data in Different Ways}\label{sec:ch05-data}

This section is effectively addressing how to use more data in different ways.

Would be nice to see a data-driven map done with the set-based framework first as an example.
May be appropriate in example for Ch 4. 

\section{Optimal Experimental Design}\label{sec:ch05-oed}

Algorithm for sequentially choosing QoI, brief discussion of tradeoffs of accuracy/precision.

\section{Sequential Inversion}\label{sec:ch05-sequential}

Outline the idea, maybe demonstrate an example. One in set-based, one in sample-based, with the quintic map???

\section{Machine-Learning Enhancements}\label{sec:ch05-ml}

how can we automate this process? Some mixture of sequential (how can we incorporate a new data point? start over? re-solve the problem? what's the cost?) approximations, maybe switching between the two approaches (set-based to reduce the parameter space, use that as the new initial). 
