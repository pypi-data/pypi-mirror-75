%% Generated by Sphinx.
\def\sphinxdocclass{report}
\documentclass[letterpaper,10pt,french]{sphinxmanual}
\ifdefined\pdfpxdimen
   \let\sphinxpxdimen\pdfpxdimen\else\newdimen\sphinxpxdimen
\fi \sphinxpxdimen=.75bp\relax

\PassOptionsToPackage{warn}{textcomp}
\usepackage[utf8]{inputenc}
\ifdefined\DeclareUnicodeCharacter
% support both utf8 and utf8x syntaxes
  \ifdefined\DeclareUnicodeCharacterAsOptional
    \def\sphinxDUC#1{\DeclareUnicodeCharacter{"#1}}
  \else
    \let\sphinxDUC\DeclareUnicodeCharacter
  \fi
  \sphinxDUC{00A0}{\nobreakspace}
  \sphinxDUC{2500}{\sphinxunichar{2500}}
  \sphinxDUC{2502}{\sphinxunichar{2502}}
  \sphinxDUC{2514}{\sphinxunichar{2514}}
  \sphinxDUC{251C}{\sphinxunichar{251C}}
  \sphinxDUC{2572}{\textbackslash}
\fi
\usepackage{cmap}
\usepackage[T1]{fontenc}
\usepackage{amsmath,amssymb,amstext}
\usepackage{babel}



\usepackage{times}
\expandafter\ifx\csname T@LGR\endcsname\relax
\else
% LGR was declared as font encoding
  \substitutefont{LGR}{\rmdefault}{cmr}
  \substitutefont{LGR}{\sfdefault}{cmss}
  \substitutefont{LGR}{\ttdefault}{cmtt}
\fi
\expandafter\ifx\csname T@X2\endcsname\relax
  \expandafter\ifx\csname T@T2A\endcsname\relax
  \else
  % T2A was declared as font encoding
    \substitutefont{T2A}{\rmdefault}{cmr}
    \substitutefont{T2A}{\sfdefault}{cmss}
    \substitutefont{T2A}{\ttdefault}{cmtt}
  \fi
\else
% X2 was declared as font encoding
  \substitutefont{X2}{\rmdefault}{cmr}
  \substitutefont{X2}{\sfdefault}{cmss}
  \substitutefont{X2}{\ttdefault}{cmtt}
\fi


\usepackage[Sonny]{fncychap}
\ChNameVar{\Large\normalfont\sffamily}
\ChTitleVar{\Large\normalfont\sffamily}
\usepackage{sphinx}

\fvset{fontsize=\small}
\usepackage{geometry}


% Include hyperref last.
\usepackage{hyperref}
% Fix anchor placement for figures with captions.
\usepackage{hypcap}% it must be loaded after hyperref.
% Set up styles of URL: it should be placed after hyperref.
\urlstyle{same}


\usepackage{sphinxmessages}
\setcounter{tocdepth}{3}
\setcounter{secnumdepth}{3}


\title{Lucterios Standard}
\date{juil. 29, 2020}
\release{2.5.0}
\author{sd-libre}
\newcommand{\sphinxlogo}{\vbox{}}
\renewcommand{\releasename}{Version}
\makeindex
\begin{document}

\ifdefined\shorthandoff
  \ifnum\catcode`\=\string=\active\shorthandoff{=}\fi
  \ifnum\catcode`\"=\active\shorthandoff{"}\fi
\fi

\pagestyle{empty}
\sphinxmaketitle
\pagestyle{plain}
\sphinxtableofcontents
\pagestyle{normal}
\phantomsection\label{\detokenize{index::doc}}



\chapter{Lucterios Standard}
\label{\detokenize{standard/index:lucterios-standard}}\label{\detokenize{standard/index::doc}}
Présentation du logiciel Lucterios Standard.


\section{Pour les utilisateurs}
\label{\detokenize{standard/index:pour-les-utilisateurs}}
L’application standard Lucterios vous permet, suite à une installation minimale du système sur votre machine, l’utilisation de modules gestions personnalisés qui répondront à vos besoins.

Les modules les plus faciles à obtenir sont les extensions open\sphinxhyphen{}source développés et maintenus par l’équipe Lusterios ou ajoutés au serveur de mises à jour par d’autres développeurs désireux de partager leur travail. Ils s’installent directement depuis la fonctionnalité de mise à jour inclus dans l’application.

Ces modules seront de plus en plus nombreux au fur et à mesure du temps et des développements. On y compte déjà plusieurs modules: rendez\sphinxhyphen{}vous sur le site de Lucterios pour en savoir plus.

Une autre solution consite à ajouter une adresse de serveur de mise à jours à orientation commercial. Les modules ainsi téléchargés nécessitent parfois une indémnité financière pour les utilisés. Consultez directement les équipes de développement qui réalisent de telles extensions sur leurs modalités.


\section{Pour les developpeurs}
\label{\detokenize{standard/index:pour-les-developpeurs}}
Lucterios est un moteur d’applications client\sphinxhyphen{}serveur, multi\sphinxhyphen{}plateforme et open\sphinxhyphen{}source reposant sur une principe de modules interdépendants.

Si vous le désirez, vous pouvez créer vos propres modules d’extensions pour répondre à vos besoins spécifiques.
Cet outil Lucterios vous permettra de créer un module et l’ensemble des actions associées à vos données: rendez\sphinxhyphen{}vous sur le site de Lucterios pour consulter les tutoriels développeurs.

De la même façon que vous avez pu apprécier des modules open\sphinxhyphen{}sources, vos extensions peuvent peut\sphinxhyphen{}être interessé d’autres personnes.
Contactez nous pour les rendres disponibles sur le serveur de mise à jours.
\begin{quote}

\sphinxhref{mailto:support@lucterios.org}{support@lucterios.org}

\sphinxurl{http://www.lucterios.org}
\end{quote}


\chapter{Lucterios contacts}
\label{\detokenize{contacts/index:lucterios-contacts}}\label{\detokenize{contacts/index::doc}}
Aide relative aux fonctionnalités de gestion de contacts moraux ou physiques.


\section{Les contact physiques (personnes physiques)}
\label{\detokenize{contacts/individual:les-contact-physiques-personnes-physiques}}\label{\detokenize{contacts/individual::doc}}
Un contact physique est une personne, homme ou femme, avec qui votre structure est en relation. Il peut s’agir d’un adhérent, d’un copropriétaire mais aussi d’un contact, personne physique, de l’un de vos interlocuteurs (exemple le commercial d’un de vos fournisseurs).
\begin{quote}

Menu \sphinxstyleemphasis{Bureautique/Adresses et Contacts/Personnes physiques}
\end{quote}


\subsection{Liste de vos contacts physiques}
\label{\detokenize{contacts/individual:liste-de-vos-contacts-physiques}}
La liste des personnes déjà enregistrées s’affiche. Étant donné que celle\sphinxhyphen{}ci peut devenir importante, il est possible de filtrer les personnes sur le nom.

Depuis cet écran, vous avez aussi la possibilité d’imprimer la liste des personnes et les étiquettes pour le courrier.

\noindent\sphinxincludegraphics{{ListIndividual}.png}


\subsection{Visualisation d’un contact physique}
\label{\detokenize{contacts/individual:visualisation-d-un-contact-physique}}
La liste des personnes physiques étant affichée, le bouton « Editer » ou un double\sphinxhyphen{}clic sur la ligne correspondante au contact, permettent de visualiser la fiche du contact.

\noindent\sphinxincludegraphics{{ShowIndividual}.png}

La fiche consultée peut être modifiée, bouton « modifier » et imprimée, bouton « Imprimer ».
Si cette personne n’est pas associée à d’autres enregistrements de l’application, vous avez la possibilité de supprimer sa fiche.

Vous pouvez également attribuer au contact physique un alias de connexion à l’application, assorti de droits et de permissions (voir Les utilisateurs).

\noindent\sphinxincludegraphics{{PermissionsIndividual}.png}


\subsection{Ajout d’un contact physique}
\label{\detokenize{contacts/individual:ajout-d-un-contact-physique}}
Depuis la liste précédente, vous avez aussi la possibilité d’ajouter une nouvelle personne à l’aide du bouton « + Créer ».

\noindent\sphinxincludegraphics{{EditIndividual}.png}


\subsection{Recherche d’un contact physique}
\label{\detokenize{contacts/individual:recherche-d-un-contact-physique}}\begin{quote}

Menu Bureautique/Adresses et Contacts/Recherche de personne physique
\end{quote}

Définissez les critères de recherche grâce à quoi seront extraites de la base toutes les fiches y satisfaisant.
Vous pourrez alors imprimer cette liste ou visualiser/modifier une fiche.

\noindent\sphinxincludegraphics{{FindIndividual}.png}

Les critères de filtre peuvent être sauvegardés pour une utilisation ultérieure.


\section{Les contacts moraux (personnes morales)}
\label{\detokenize{contacts/legal_entity:les-contacts-moraux-personnes-morales}}\label{\detokenize{contacts/legal_entity::doc}}
Entreprises, associations, établissements publics, …, les personnes morales sont des groupements de personnes dotées de la personnalité juridique. Elles peuvent être composées d’une ou de plusieurs personnes, individus ou personnes morales.
Pour chacune des personnes morales avec qui votre structure est en contact, une fiche peut être tenue sous \sphinxstyleemphasis{Diacamma}.
\begin{quote}

Menu \sphinxstyleemphasis{Bureautique/Adresses et Contacts/Personnes morales}
\end{quote}


\subsection{Liste de vos contacts moraux}
\label{\detokenize{contacts/legal_entity:liste-de-vos-contacts-moraux}}
Vous pouvez consulter la liste des structures pour lesquelles une fiche a déjà été enregistrée.
Chaque contact moral est associé à une catégorie grâce à quoi vous pouvez filtrer la liste des contacts sur le type de structures.

\noindent\sphinxincludegraphics{{ListLegalEntity}.png}

Depuis cet écran, vous avez aussi la possibilité d’imprimer la liste des structures avec le bouton « Liste » et pouvez imprimer les étiquettes pour le courrier.

Une nouvelle fiche est ouverte à l’aide du bouton « + Créer » et toute fiche peut être supprimée, à la condition de ne pas être associée à un autre enregistrement de votre base (exemple une facture saisie).


\subsection{Visualisation d’un contact moral}
\label{\detokenize{contacts/legal_entity:visualisation-d-un-contact-moral}}
Depuis la liste précédente, la fiche d’une structure peut être visualisée à l’aide du bouton « Editer » ou d’un double\sphinxhyphen{}clic sur la ligne correspondante au contact.

\noindent\sphinxincludegraphics{{ShowLegalEntity}.png}

Cette fiche peut ensuite être imprimée avec le bouton du même nom.


\subsection{Modifier un contact moral}
\label{\detokenize{contacts/legal_entity:modifier-un-contact-moral}}
La fiche étant toujours à l’écran, utilisez le bouton « Modifier » pour y apporter toute correction.

\noindent\sphinxincludegraphics{{EditLegalEntity}.png}


\subsection{Responsables d’un contact moral}
\label{\detokenize{contacts/legal_entity:responsables-d-un-contact-moral}}
Vous avez la possibilité d’associer une personne physique à un contact moral : onglet « Membres » et bouton « + Ajouter ».
Sélectionnez la personne physique. Si elle n’est pas répertorié dans votre base, vous avez la possibilité d’y pourvoir.
Tout nouveau membre peut être assorti d’une fonction.

\noindent\sphinxincludegraphics{{ResponsabilityLegalEntity}.png}


\subsection{Recherche d’un contact moral}
\label{\detokenize{contacts/legal_entity:recherche-d-un-contact-moral}}
Le menu \sphinxstyleemphasis{Bureautique/Adresses et Contacts/Recherche de personne morale} vous permet d’extraire de votre base les personnes morales satisfaisant aux critères saisis. Ces critères peuvent être sauvegardés pour une utilisation ultérieure.

\noindent\sphinxincludegraphics{{FindLegalEntity}.png}


\section{Configuration et paramétrage}
\label{\detokenize{contacts/configuration:configuration-et-parametrage}}\label{\detokenize{contacts/configuration::doc}}
Dans le menu \sphinxstyleemphasis{Administration/Modules (conf.)} vous avez à votre disposition des outils pour configurer la gestion des contacts.


\subsection{Configuration des contacts}
\label{\detokenize{contacts/configuration:configuration-des-contacts}}
Dans cet écran, vous avez la possibilité de créer ou de modifier la liste des fonctions et des responsabilités que peuvent avoir les contacts  associés aux personnes physiques et morales avec lesquelles vous êtes en relation. Vous pouvez aussi créer ou modifier les types de structures pour vous aider à classer vos personnes morales.

Il se peut que vous ayez besoin de préciser des informations supplémentaires pour vos différents contacts. Vous avez ici la possibilité d’ajouter des champs personnalisés pour chaque type de contacts. Pour ajouter un champ, vous devez simplement préciser le modèle de contact auquel il s’applique, indiquer son titre ainsi que définir son type et préciser si le champ est multiligne ou non.

Cinq types sont possibles :
\begin{itemize}
\item {} 
chaîne de texte

\item {} 
nombre entier

\item {} 
nombre à virgule (réel)

\item {} 
valeur Oui/Non (booléen)

\item {} 
sélection dans une liste (énumération)

\end{itemize}

Dans le cas de l’énumération, vous devez définir la liste des valeurs possibles (mots) séparées par un point\sphinxhyphen{}virgule.


\subsection{Codes postaux/villes}
\label{\detokenize{contacts/configuration:codes-postaux-villes}}
En saisie, l’outil va automatiquement rechercher la ville (ou les villes) associée(s) au code postal que vous entrerez afin de vous faciliter la saisie de vos contacts.
Dans cet écran, vous pouvez ajouter les codes postaux manquants.
Par défaut, les codes postaux français et suisses sont insérés.


\chapter{Lucterios messagerie}
\label{\detokenize{mailing/index:lucterios-messagerie}}\label{\detokenize{mailing/index::doc}}
Aide relative aux fonctionnalités de courier, de publipostage et de SMS.


\section{Publipostage}
\label{\detokenize{mailing/mailing:publipostage}}\label{\detokenize{mailing/mailing::doc}}\begin{quote}

Menu \sphinxstyleemphasis{Bureautique/Publipostage/Messages courriel}
\end{quote}


\subsection{Création d’un message}
\label{\detokenize{mailing/mailing:creation-d-un-message}}
Bouton « + Ajouter »

Une fois votre message rédigé, validez\sphinxhyphen{}le en cliquant sur « Ok ». En ouvrant l’onglet « Destinataires », vous allez pouvoir saisir le ou les critères permettant de filtrer vos contacts. Le résultat de la requête est affichée à l’écran. Vous pouvez l’affiner et sauvegarder vos critères pour une utilisation ultérieure.

Votre requête prête, cliquez sur « validée ». D’autres requêtes peuvent être ajoutées à la première, leurs résultats se cumulant.
Au moment où votre courrier sera généré, vos requêtes seront de nouveau exécutées, grâce à quoi votre base peut être mise à jour avant \sphinxstyleemphasis{Validation et transmission} des courriers si vous constatez qu’un contact est absent du résultat des requêtes.

Il est également possible de joindre à votre message un ou plusieurs documents sauvegardés dans le \sphinxstyleemphasis{Gestionnaire de documents}. Pour cela, ouvrez l’onglet « Documents ».

L’option \sphinxstyleemphasis{document(s) ajouté(s) via liens dans le message} permet d’ajouter un ensemble de liens de partage vers vos documents (et non plus comme pièces jointes). Cela permet la transmission de documents de taille importante ou qui risqueraient d’être supprimés par certains gestionnaires de courriel.

\noindent\sphinxincludegraphics{{mailing}.png}


\subsection{Validation \& transmission}
\label{\detokenize{mailing/mailing:validation-transmission}}\begin{description}
\item[{Une fois le message validé vous pouvez :}] \leavevmode\begin{itemize}
\item {} 
Soit générer une sortie PDF de l’ensemble des lettres à envoyer, personnalisées avec l’en\sphinxhyphen{}tête de chaque contact

\item {} 
Soit envoyer votre message par courriel, si celui\sphinxhyphen{}ci est correctement configuré. Bien sur, dans ce cas, seuls les contacts possédant une adresse électronique seront impactés par cet envoi.

\end{itemize}

\end{description}

De plus, dans le cas d’un envoi par courriel, vous pouvez consulter le rapport de transmission. Celui\sphinxhyphen{}ci vous indique les courriels envoyés et les éventuelles erreurs d’acheminement.

Si votre logiciel est accessible depuis internet, vous pouvez également consulter le nombre de fois que le destinataire a consulté ce message.
Ce mécanisme se base sur l’acceptation, par votre destinataire des images distantes présentent dans le message.

\noindent\sphinxincludegraphics{{transmission}.png}


\section{Envoi de SMS}
\label{\detokenize{mailing/sms:envoi-de-sms}}\label{\detokenize{mailing/sms::doc}}\begin{quote}

Menu \sphinxstyleemphasis{Bureautique/Publipostage/Messages SMS}
\end{quote}


\subsection{Création d’un message}
\label{\detokenize{mailing/sms:creation-d-un-message}}
L’envoi de SMS se fait via une interface proche de celle d’envoie de couriels.

Notons les différences:
\begin{itemize}
\item {} 
Spécificité de création de message
\begin{quote}

Vous devez précisez quel champ (tel1, tel2 ou un champ textuel personalisé) vous souhaitez utiliser pour envoyer vos messages.
Notez qu’il est possible d’envoyer à plusieurs numéro pour un même contact.
\end{quote}

\item {} 
La taille du message de celui\sphinxhyphen{}ci s’affiche.
\begin{quote}

Notez que certain caractère, comme les accents, peuvent compter double.
\end{quote}

\item {} 
Il n’est pas possible d’ajouter de document joint comme pour les courriels.

\end{itemize}


\subsection{Validation \& transmission}
\label{\detokenize{mailing/sms:validation-transmission}}
Une fois le message validé vous pouvez l’envoyer par SMS, si celui\sphinxhyphen{}ci est correctement configuré.
Bien sur, dans ce cas, seuls les contacts possédant une numéro de téléphone valide (voir configuration) seront impactés par cet envoi.

De plus, vous pouvez consulter le rapport de transmission.
Celui\sphinxhyphen{}ci vous indique les SMS envoyés et les éventuelles erreurs d’acheminement.


\section{Configuration de la messagerie}
\label{\detokenize{mailing/configuration:configuration-de-la-messagerie}}\label{\detokenize{mailing/configuration::doc}}\begin{quote}

Menu Administration/Modules (conf.)/Paramètres de courrier \& SMS
\end{quote}

Vous pouvez configurer ici des réglages pour l’envoi de couriel et de SMS.


\subsection{Configuration du couriel}
\label{\detokenize{mailing/configuration:configuration-du-couriel}}
Le serveur SMTP permettra au logiciel d’envoyer directement des messages à vos contacts.
Configurez donc ici les règlages de votre serveur.
Vous pouvez également préciser un \sphinxstyleemphasis{Fichier privé DKIM} et \sphinxstyleemphasis{Sélecteur DKIM} afin de signer vos envois de courriel.
Les paramètres \sphinxstyleemphasis{durée (en min) d’un lot de courriel} et \sphinxstyleemphasis{nombre de courriels par lot} sont utilisés pour l’envoie des messages en publipostage.

Un bouton \sphinxstyleemphasis{Envoyer} permet de tester vos règlages en envoyant un courriel de test à un destinataire choisi.
Il existe des outils permettant de vérifier si vos messages respectent des règles afin d’éviter d’être considéré comme des “pourriel”.
En autre, l’outil \sphinxurl{https://www.mail-tester.com} (gratuit jusqu’à 3 fois par jour) vous permet, en envoyant un message à l’adresse précisée, de vous établir une note de confiance.

Vous pouvez, entre autre, envoyer d’un nouveau mot de passe de connexion.
N’oubliez pas alors de préciser un petit message d’explication via le paramètre \sphinxstyleemphasis{Message de confirmation de connexion}.


\subsection{Configuration du SMS}
\label{\detokenize{mailing/configuration:configuration-du-sms}}
En configurant un fournisseur de SMS, vous pourrez alors envoyer des SMS à vos contacts.

Pour chaque fournisseur, vous devrez préciser le champ « Expression d’analyse de numéro (SMS) ».
Il correspond à une expression régulière afin de savoir comment transformer un numéro de téléphone en numéro international.
Par défaut, il correspond au numéro français: \sphinxstyleemphasis{\textasciicircum{}0({[}67{]}{[}0\sphinxhyphen{}9{]}\{8\})\$|+33\{0\}}
\begin{quote}

Il vérifie que le numéro comporte 10 chiffres.
Il commance par “06” ou “07”.
Il remplacera le “0” par “+33”.
\end{quote}

Suivant les besoins et les demandes, d’autres fournisseurs pourront être ajoutés à l’avenir.


\subsubsection{Mailjet SMS}
\label{\detokenize{mailing/configuration:mailjet-sms}}
Mailjet (\sphinxurl{https://www.mailjet.com/}) est une entreprise française qui propose, entre autres, d’envoyer des SMS.

Pour utiliser ce fournisseur, vous devez créer un compte sur leur site web.

Rendez vous ensuite sur leur portail SMS (\sphinxurl{https://app.mailjet.com/sms}) où vous pourrez configurer votre accès au SMS:
\begin{itemize}
\item {} 
Générer votre token d’accès

\item {} 
Créditer votre solde de SMS prépayé

\end{itemize}

Notez que le coùt d’un SMS dépend de sa taille et de sa destination.

Vous devrez ensuite indiquer dans « Options pour fournisseur SMS » votre « api token » généré précédement
ainsi qu’un « alias » qui sera l’identifiant présent dans le SMS envoyé.

Une fois votre configuration terminer, un bouton \sphinxstyleemphasis{Envoyer} permet de tester vos règlages en envoyant un SMS de contrôle à un numéro choisi.


\chapter{Lucterios documents}
\label{\detokenize{documents/index:lucterios-documents}}\label{\detokenize{documents/index::doc}}
Aide relative aux fonctionnalités de gestion documentaire.


\section{Fichiers partagés}
\label{\detokenize{documents/shared_document:fichiers-partages}}\label{\detokenize{documents/shared_document::doc}}

\subsection{Liste des documents}
\label{\detokenize{documents/shared_document:liste-des-documents}}
Pour retrouver plus aisément vos documents, sous \sphinxstyleemphasis{Diacamma}, ceux\sphinxhyphen{}ci peuvent être enregistrés dans des dossiers et des sous\sphinxhyphen{}dossiers du gestionnaire de documents.
Chaque dossier est assorti d’une description et d’informations relatives à la dernière modification.
\begin{quote}

Menu \sphinxstyleemphasis{Bureautique/Gestion de fichiers et de documents/Documents}
\end{quote}

\noindent\sphinxincludegraphics{{listfiles}.png}

En utilisant le bouton « + Dossier », il est possible de créer un nouveau dossier.
Après l’avoir sélectionné, vous pouvez aussi en supprimer un et son contenu, à l’aide du bouton « \sphinxhyphen{}« .
Le bouton situé juste au dessus de « + Dossier » permet, lui, d’éditer les propriétés du dossier actif et de les modifier, à la condition de disposer des droits pour cela.

\sphinxstyleemphasis{Diacamma} mémorise l’utilisateur et la date de création de tout document ainsi que les informations relatives à la dernière modification. Affichez le contenu d’un dossier après l’avoir sélectionné et avoir cliqué sur le bouton « Editer » ou faites un double\sphinxhyphen{}clic sur la ligne correspondante au dossier.

\noindent\sphinxincludegraphics{{listdoc}.png}

Le retour au dossier\sphinxhyphen{}parent est possible grâce au bouton « \textless{} »
Suivant vos permissions, vous pouvez ajouter un document dans un dossier.

Affichez la fiche d’un document à l’aide d’un double\sphinxhyphen{}clic sur son nom.

\noindent\sphinxincludegraphics{{showdoc}.png}
\begin{description}
\item[{Depuis cette fiche, il vous est possible :}] \leavevmode\begin{itemize}
\item {} 
de le modifier en l’important de nouveau

\item {} 
de générer un lien de téléchargement. Ce lien web peut être transmis à une personne n’ayant aucun droit d’accès à votre logiciel afin qu’elle puisse télécharger le document

\end{itemize}

\end{description}

\sphinxstylestrong{Attention:} Votre instance doit être accessible sur internet pour que ce lien puisse fonctionner.


\subsection{Recherche de documents}
\label{\detokenize{documents/shared_document:recherche-de-documents}}\begin{quote}

Menu \sphinxstyleemphasis{Bureautique/Gestion documentaire/Recherche de document}
\end{quote}

Saisissez les critères de recherche de documents et validez. Ces critères sont sauvegardables pour une utilisation ultérieure.

Diacamma parcourera tous les dossiers du gestionnaire de documents afin d’en extraire la liste de ceux satisfaisant aux critères saisis.


\section{Editeur de documents}
\label{\detokenize{documents/editor:editeur-de-documents}}\label{\detokenize{documents/editor::doc}}
Il est possible de configurer l’outil afin de pouvoir éditer certains documents directement via l’interface « en ligne ».

Des outils d’édition, libres et gratuits, sont actuellement configurables afin de les utiliser pour consulter et modifier des documents.

\sphinxstylestrong{Note :} Ces outils sont gérés par des équipes complètement différentes, il se peut que certains de leurs comportements ne correspondent pas à vos attentes.


\subsection{Etherpad}
\label{\detokenize{documents/editor:etherpad}}
Editeur pour document textuel.
\begin{description}
\item[{Site Web}] \leavevmode
\sphinxurl{https://etherpad.org/}

\item[{Installation}] \leavevmode
Le tutoriel de framasoft explique bien comment l’installer
\sphinxurl{https://framacloud.org/fr/cultiver-son-jardin/etherpad.html}

\item[{Configurer}] \leavevmode
Editer le fichier « settings.py » contenu dans le répertoire de votre instance.
Ajouter et adapter la ligne ci\sphinxhyphen{}dessous:
\begin{itemize}
\item {} 
url : adresse d’accès d’Etherpad

\item {} 
apikey : contenu de la clef de sécurité (fichier APIKEY.txt contenu dans l’installation d’etherpad)

\end{itemize}

\end{description}

\begin{sphinxVerbatim}[commandchars=\\\{\}]
\PYG{c+c1}{\PYGZsh{} extra}
\PYG{n}{ETHERPAD} \PYG{o}{=} \PYG{p}{\PYGZob{}}\PYG{l+s+s1}{\PYGZsq{}}\PYG{l+s+s1}{url}\PYG{l+s+s1}{\PYGZsq{}}\PYG{p}{:} \PYG{l+s+s1}{\PYGZsq{}}\PYG{l+s+s1}{http://localhost:9001}\PYG{l+s+s1}{\PYGZsq{}}\PYG{p}{,} \PYG{l+s+s1}{\PYGZsq{}}\PYG{l+s+s1}{apikey}\PYG{l+s+s1}{\PYGZsq{}}\PYG{p}{:} \PYG{l+s+s1}{\PYGZsq{}}\PYG{l+s+s1}{jfks5dsdS65lfGHsdSDQ4fsdDG4lklsdq6Gfs4Gsdfos8fs}\PYG{l+s+s1}{\PYGZsq{}}\PYG{p}{\PYGZcb{}}
\end{sphinxVerbatim}
\begin{description}
\item[{Usage}] \leavevmode\begin{description}
\item[{Dans le gestionnaire de documents, vous avez plusieurs actions qui apparaissent alors}] \leavevmode\begin{itemize}
\item {} 
Un bouton « + Fichier » vous permettant de créer un document txt ou html

\item {} 
Un bouton « Editeur » pour ouvrir l’éditeur Etherpad.

\end{itemize}

\end{description}

\end{description}

\noindent\sphinxincludegraphics{{etherpad}.png}


\subsection{Ethercalc}
\label{\detokenize{documents/editor:ethercalc}}
Editeur pour tableau de calcul.
\begin{description}
\item[{Site Web}] \leavevmode
\sphinxurl{https://ethercalc.net/}

\item[{Installation}] \leavevmode
Sur le site de l’éditeur, une petit explication indique comment l’installer.

\item[{Configurer}] \leavevmode
Editer le fichier « settings.py » contenu dans le répertoire de votre instance.
Ajouter et adapter la ligne ci\sphinxhyphen{}dessous:
\begin{itemize}
\item {} 
url : adresse d’accès d’Ethercal

\end{itemize}

\end{description}

\begin{sphinxVerbatim}[commandchars=\\\{\}]
\PYG{c+c1}{\PYGZsh{} extra}
\PYG{n}{ETHERCALC} \PYG{o}{=} \PYG{p}{\PYGZob{}}\PYG{l+s+s1}{\PYGZsq{}}\PYG{l+s+s1}{url}\PYG{l+s+s1}{\PYGZsq{}}\PYG{p}{:} \PYG{l+s+s1}{\PYGZsq{}}\PYG{l+s+s1}{http://localhost:8000}\PYG{l+s+s1}{\PYGZsq{}}\PYG{p}{\PYGZcb{}}
\end{sphinxVerbatim}
\begin{description}
\item[{Usage}] \leavevmode\begin{description}
\item[{Dans le gestionnaire de documents, vous avez plusieurs actions qui apparaissent alors}] \leavevmode\begin{itemize}
\item {} 
Un bouton « + Fichier » vous permettant de créer un document csv, ods ou xmlx

\item {} 
Un bouton « Editeur » pour ouvrir l’éditeur Ethercalc.

\end{itemize}

\end{description}

\end{description}

\noindent\sphinxincludegraphics{{ethercalc}.png}


\section{Configurer les dossiers}
\label{\detokenize{documents/configuration:configurer-les-dossiers}}\label{\detokenize{documents/configuration::doc}}
Pour votre gestion documentaire, vous disposez d’un ensemble d’outils.
\begin{quote}

Menu \sphinxstyleemphasis{Administration/Module (conf)/Dossiers}
\end{quote}

A l’écran la liste des dossiers existants s’affiche. Vous avez la possibilité d’en créer de nouveaux et de modifier les paramètres des dossiers déjà présents.

\noindent\sphinxincludegraphics{{configuration}.png}

En définissant judicieusement un dossier comme sous\sphinxhyphen{}dossier d’un dossier\sphinxhyphen{}parent, vous pouvez  mettre en place une arborescence de dossiers respectueuse de votre plan de classement.

Pour chaque dossier, vous pouvez aussi définir les droits des groupes d’utilisateurs, que cela soit pour la visualisation ou la modification des fichiers qui sont/seront enregistrés dans le dossier.
Ainsi, les utilisateurs appartenant aux groupes de visualisation pourront seulement consulter les documents contenus dans ce dossier. Les utilisateurs appartenant aux groupes de modification pourront, eux, les modifier ou les supprimer.

Le bouton « Extraire » permet de réaliser la sauvegarde d’un dossier et de son contenu sous forme d’une archive au format zip.
Quand au bouton « Importer », il permet d’importer une archive au format zip qui sera décompressée automatiquement dans le dossier de destination spécifié.


\chapter{Coeur Lucterios}
\label{\detokenize{CORE/index:coeur-lucterios}}\label{\detokenize{CORE/index::doc}}
Aide relative aux fonctionnalités générales de cet outil de gestion.


\section{Mot de passe}
\label{\detokenize{CORE/password:mot-de-passe}}\label{\detokenize{CORE/password::doc}}\begin{quote}

Menu \sphinxstyleemphasis{Général/Mot de passe}
\end{quote}

Vous pouvez changer le mot de passe d’accès de l’utilisateur courant.

\noindent\sphinxincludegraphics{{password}.png}

Pour plus de sécurité, nous vous conseillons d’utiliser un mot de passe comprenant des lettres et des chiffres et ne constituant pas un mot compréhensible.


\section{Les groupes}
\label{\detokenize{CORE/groups:les-groupes}}\label{\detokenize{CORE/groups::doc}}\begin{quote}

Menu \sphinxstyleemphasis{Administration/Gestion des Droits/Les groupes}
\end{quote}

Créez, modifiez ou supprimez un groupe de droits.

\noindent\sphinxincludegraphics{{group}.png}

Un groupe de droits permet de définir les autorisations (accès à certaines fonctionnalités du logiciel) qui sont consenties aux utilisateurs de l’application rattachés à celui\sphinxhyphen{}ci.

\noindent\sphinxincludegraphics{{group_modify}.png}


\section{Les utilisateurs}
\label{\detokenize{CORE/users:les-utilisateurs}}\label{\detokenize{CORE/users::doc}}\begin{quote}

Menu \sphinxstyleemphasis{Administration/Gestion des Droits/Les utilisateurs}
\end{quote}

Créez, modifiez ou désactivez un utilisateur de l’application. Chacun bénéficie d’un droit de connexion à \sphinxstyleemphasis{diacamma} dont vous définissez l’étendue.

\noindent\sphinxincludegraphics{{users}.png}

Depuis cette liste, vous pouvez créer ou modifier un utilisateur : son alias, son nom et son mot de passe.

\noindent\sphinxincludegraphics{{user_info}.png}

Vous pouvez aussi l’inscrire dans des groupes, lui accorder des permissions suplémentaires afin de définir son niveau d’accès au logiciel. Vous pouvez aussi désactiver un utilisateur pour lui interdire l’accès à l’application.

\noindent\sphinxincludegraphics{{user_permissions}.png}


\section{L’architecture du logiciel}
\label{\detokenize{CORE/architecture:l-architecture-du-logiciel}}\label{\detokenize{CORE/architecture::doc}}
Depuis le commencement de ce logiciel, les développeurs ont voulu que cette application puisse avoir une architecture ouverte permettant des évolutions les plus larges.



\renewcommand{\indexname}{Index}
\printindex
\end{document}