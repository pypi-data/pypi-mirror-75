%% Generated by Sphinx.
\def\sphinxdocclass{report}
\documentclass[letterpaper,10pt,english]{sphinxmanual}
\ifdefined\pdfpxdimen
   \let\sphinxpxdimen\pdfpxdimen\else\newdimen\sphinxpxdimen
\fi \sphinxpxdimen=.75bp\relax
%% turn off hyperref patch of \index as sphinx.xdy xindy module takes care of
%% suitable \hyperpage mark-up, working around hyperref-xindy incompatibility
\PassOptionsToPackage{hyperindex=false}{hyperref}
\PassOptionsToPackage{svgnames}{xcolor}
\PassOptionsToPackage{warn}{textcomp}
\usepackage[utf8]{inputenc}
\ifdefined\DeclareUnicodeCharacter
% support both utf8 and utf8x syntaxes
  \ifdefined\DeclareUnicodeCharacterAsOptional
    \def\sphinxDUC#1{\DeclareUnicodeCharacter{"#1}}
  \else
    \let\sphinxDUC\DeclareUnicodeCharacter
  \fi
  \sphinxDUC{00A0}{\nobreakspace}
  \sphinxDUC{2500}{\sphinxunichar{2500}}
  \sphinxDUC{2502}{\sphinxunichar{2502}}
  \sphinxDUC{2514}{\sphinxunichar{2514}}
  \sphinxDUC{251C}{\sphinxunichar{251C}}
  \sphinxDUC{2572}{\textbackslash}
\fi
\usepackage{cmap}
\usepackage[LGR,X2,T1]{fontenc}
\usepackage{amsmath,amssymb,amstext}
\usepackage{babel}
\usepackage{substitutefont}
\usepackage[Xtwo]{sphinxcyrillic}

\usepackage[sc]{mathpazo}
\usepackage[scaled]{helvet}
\usepackage{courier}
\substitutefont{LGR}{\rmdefault}{cmr}
\substitutefont{LGR}{\sfdefault}{cmss}
\substitutefont{LGR}{\ttdefault}{cmtt}
\substitutefont{X2}{\rmdefault}{cmr}
\substitutefont{X2}{\sfdefault}{cmss}
\substitutefont{X2}{\ttdefault}{cmtt}

\usepackage{textalpha}
\usepackage[Bjarne]{fncychap}
\usepackage{sphinx}

\fvset{fontsize=auto}
\usepackage{geometry}


% Include hyperref last.
\usepackage{hyperref}
% Fix anchor placement for figures with captions.
\usepackage{hypcap}% it must be loaded after hyperref.
% Set up styles of URL: it should be placed after hyperref.
\urlstyle{same}


\usepackage{sphinxmessages}
\setcounter{tocdepth}{1}

\DeclareUnicodeCharacter{229E}{\ensuremath{\boxplus}}

\title{Sphinx Documentation}
\date{Jul 06, 2020}
\release{3.2.0}
\author{Georg Brandl}
\newcommand{\sphinxlogo}{\sphinxincludegraphics{sphinx.png}\par}
\renewcommand{\releasename}{Release}
\makeindex
\begin{document}

\pagestyle{empty}
\sphinxmaketitle
\pagestyle{plain}
\sphinxtableofcontents
\pagestyle{normal}
\phantomsection\label{\detokenize{contents::doc}}



\chapter{Using Sphinx}
\label{\detokenize{usage/index:using-sphinx}}\label{\detokenize{usage/index::doc}}
This guide serves to demonstrate how one can get started with Sphinx and covers
everything from installing Sphinx and configuring your first Sphinx project to
using some of the advanced features Sphinx provides out\sphinxhyphen{}of\sphinxhyphen{}the\sphinxhyphen{}box. If you are
looking for guidance on extending Sphinx, refer to \DUrole{xref,std,std-doc}{/development/index}.


\section{Internationalization}
\label{\detokenize{usage/advanced/intl:internationalization}}\label{\detokenize{usage/advanced/intl:intl}}\label{\detokenize{usage/advanced/intl::doc}}
\DUrole{versionmodified,added}{New in version 1.1.}

Complementary to translations provided for Sphinx\sphinxhyphen{}generated messages such as
navigation bars, Sphinx provides mechanisms facilitating \sphinxstyleemphasis{document} translations
in itself.  See the \DUrole{xref,std,std-ref}{intl\sphinxhyphen{}options} for details on configuration.

\begin{figure}[htbp]
\centering
\capstart

\noindent\sphinxincludegraphics[width=1.000\linewidth]{{translation}.png}
\caption{Workflow visualization of translations in Sphinx.  (The figure is created by
\sphinxhref{http://plantuml.com}{plantuml}\sphinxfootnotemark[3].)}\label{\detokenize{usage/advanced/intl:id5}}\end{figure}
%
\begin{footnotetext}[3]\sphinxAtStartFootnote
\sphinxnolinkurl{http://plantuml.com}
%
\end{footnotetext}\ignorespaces 
\begin{sphinxShadowBox}
\begin{itemize}
\item {} 
\phantomsection\label{\detokenize{usage/advanced/intl:id6}}{\hyperref[\detokenize{usage/advanced/intl:sphinx-internationalization-details}]{\sphinxcrossref{Sphinx internationalization details}}}

\item {} 
\phantomsection\label{\detokenize{usage/advanced/intl:id7}}{\hyperref[\detokenize{usage/advanced/intl:translating-with-sphinx-intl}]{\sphinxcrossref{Translating with sphinx\sphinxhyphen{}intl}}}
\begin{itemize}
\item {} 
\phantomsection\label{\detokenize{usage/advanced/intl:id8}}{\hyperref[\detokenize{usage/advanced/intl:quick-guide}]{\sphinxcrossref{Quick guide}}}

\item {} 
\phantomsection\label{\detokenize{usage/advanced/intl:id9}}{\hyperref[\detokenize{usage/advanced/intl:translating}]{\sphinxcrossref{Translating}}}

\item {} 
\phantomsection\label{\detokenize{usage/advanced/intl:id10}}{\hyperref[\detokenize{usage/advanced/intl:update-your-po-files-by-new-pot-files}]{\sphinxcrossref{Update your po files by new pot files}}}

\end{itemize}

\item {} 
\phantomsection\label{\detokenize{usage/advanced/intl:id11}}{\hyperref[\detokenize{usage/advanced/intl:using-transifex-service-for-team-translation}]{\sphinxcrossref{Using Transifex service for team translation}}}

\item {} 
\phantomsection\label{\detokenize{usage/advanced/intl:id12}}{\hyperref[\detokenize{usage/advanced/intl:contributing-to-sphinx-reference-translation}]{\sphinxcrossref{Contributing to Sphinx reference translation}}}

\end{itemize}
\end{sphinxShadowBox}


\subsection{Sphinx internationalization details}
\label{\detokenize{usage/advanced/intl:sphinx-internationalization-details}}
\sphinxstylestrong{gettext} %
\begin{footnote}[1]\sphinxAtStartFootnote
See the \sphinxhref{https://www.gnu.org/software/gettext/manual/gettext.html\#Introduction}{GNU gettext utilities}\sphinxfootnotemark[13]
for details on that software suite.
%
\end{footnote}%
\begin{footnotetext}[13]\sphinxAtStartFootnote
\sphinxnolinkurl{https://www.gnu.org/software/gettext/manual/gettext.html\#Introduction}
%
\end{footnotetext}\ignorespaces  is an established standard for internationalization and
localization.  It naively maps messages in a program to a translated string.
Sphinx uses these facilities to translate whole documents.

Initially project maintainers have to collect all translatable strings (also
referred to as \sphinxstyleemphasis{messages}) to make them known to translators.  Sphinx extracts
these through invocation of \sphinxcode{\sphinxupquote{sphinx\sphinxhyphen{}build \sphinxhyphen{}b gettext}}.

Every single element in the doctree will end up in a single message which
results in lists being equally split into different chunks while large
paragraphs will remain as coarsely\sphinxhyphen{}grained as they were in the original
document.  This grants seamless document updates while still providing a little
bit of context for translators in free\sphinxhyphen{}text passages.  It is the maintainer’s
task to split up paragraphs which are too large as there is no sane automated
way to do that.

After Sphinx successfully ran the
\sphinxcode{\sphinxupquote{MessageCatalogBuilder}} you will find a
collection of \sphinxcode{\sphinxupquote{.pot}} files in your output directory.  These are \sphinxstylestrong{catalog
templates} and contain messages in your original language \sphinxstyleemphasis{only}.

They can be delivered to translators which will transform them to \sphinxcode{\sphinxupquote{.po}} files
— so called \sphinxstylestrong{message catalogs} — containing a mapping from the original
messages to foreign\sphinxhyphen{}language strings.

\sphinxstyleemphasis{gettext} compiles them into a binary format known as \sphinxstylestrong{binary catalogs}
through \sphinxstyleliteralstrong{\sphinxupquote{msgfmt}} for efficiency reasons.  If you make these files
discoverable with \sphinxcode{\sphinxupquote{locale\_dirs}} for your \sphinxcode{\sphinxupquote{language}}, Sphinx
will pick them up automatically.

An example: you have a document \sphinxcode{\sphinxupquote{usage.rst}} in your Sphinx project.  The
\sphinxstyleemphasis{gettext} builder will put its messages into \sphinxcode{\sphinxupquote{usage.pot}}.  Imagine you have
Spanish translations %
\begin{footnote}[2]\sphinxAtStartFootnote
Because nobody expects the Spanish Inquisition!
%
\end{footnote} stored in \sphinxcode{\sphinxupquote{usage.po}} — for your builds to
be translated you need to follow these instructions:
\begin{itemize}
\item {} 
Compile your message catalog to a locale directory, say \sphinxcode{\sphinxupquote{locale}}, so it
ends up in \sphinxcode{\sphinxupquote{./locale/es/LC\_MESSAGES/usage.mo}} in your source directory
(where \sphinxcode{\sphinxupquote{es}} is the language code for Spanish.)

\begin{sphinxVerbatim}[commandchars=\\\{\}]
\PYG{n}{msgfmt} \PYG{l+s+s2}{\PYGZdq{}}\PYG{l+s+s2}{usage.po}\PYG{l+s+s2}{\PYGZdq{}} \PYG{o}{\PYGZhy{}}\PYG{n}{o} \PYG{l+s+s2}{\PYGZdq{}}\PYG{l+s+s2}{locale/es/LC\PYGZus{}MESSAGES/usage.mo}\PYG{l+s+s2}{\PYGZdq{}}
\end{sphinxVerbatim}

\item {} 
Set \sphinxcode{\sphinxupquote{locale\_dirs}} to \sphinxcode{\sphinxupquote{{[}"locale/"{]}}}.

\item {} 
Set \sphinxcode{\sphinxupquote{language}} to \sphinxcode{\sphinxupquote{es}} (also possible via
\sphinxcode{\sphinxupquote{\sphinxhyphen{}D}}).

\item {} 
Run your desired build.

\end{itemize}


\subsection{Translating with sphinx\sphinxhyphen{}intl}
\label{\detokenize{usage/advanced/intl:translating-with-sphinx-intl}}

\subsubsection{Quick guide}
\label{\detokenize{usage/advanced/intl:quick-guide}}
\sphinxhref{https://pypi.org/project/sphinx-intl/}{sphinx\sphinxhyphen{}intl}%
\begin{footnote}[4]\sphinxAtStartFootnote
\sphinxnolinkurl{https://pypi.org/project/sphinx-intl/}
%
\end{footnote} is a useful tool to work with Sphinx translation flow.  This
section describe an easy way to translate with \sphinxstyleemphasis{sphinx\sphinxhyphen{}intl}.
\begin{enumerate}
\sphinxsetlistlabels{\arabic}{enumi}{enumii}{}{.}%
\item {} 
Install \sphinxhref{https://pypi.org/project/sphinx-intl/}{sphinx\sphinxhyphen{}intl}%
\begin{footnote}[5]\sphinxAtStartFootnote
\sphinxnolinkurl{https://pypi.org/project/sphinx-intl/}
%
\end{footnote}.

\begin{sphinxVerbatim}[commandchars=\\\{\}]
\PYG{g+gp}{\PYGZdl{}} pip install sphinx\PYGZhy{}intl
\end{sphinxVerbatim}

\item {} 
Add configurations to \sphinxcode{\sphinxupquote{conf.py}}.

\begin{sphinxVerbatim}[commandchars=\\\{\}]
\PYG{n}{locale\PYGZus{}dirs} \PYG{o}{=} \PYG{p}{[}\PYG{l+s+s1}{\PYGZsq{}}\PYG{l+s+s1}{locale/}\PYG{l+s+s1}{\PYGZsq{}}\PYG{p}{]}   \PYG{c+c1}{\PYGZsh{} path is example but recommended.}
\PYG{n}{gettext\PYGZus{}compact} \PYG{o}{=} \PYG{k+kc}{False}     \PYG{c+c1}{\PYGZsh{} optional.}
\end{sphinxVerbatim}

This case\sphinxhyphen{}study assumes that BUILDDIR is set to \sphinxcode{\sphinxupquote{\_build}},
\sphinxcode{\sphinxupquote{locale\_dirs}} is set to \sphinxcode{\sphinxupquote{locale/}} and \sphinxcode{\sphinxupquote{gettext\_compact}}
is set to \sphinxcode{\sphinxupquote{False}} (the Sphinx document is already configured as such).

\item {} 
Extract translatable messages into pot files.

\begin{sphinxVerbatim}[commandchars=\\\{\}]
\PYG{g+gp}{\PYGZdl{}} make gettext
\end{sphinxVerbatim}

The generated pot files will be placed in the \sphinxcode{\sphinxupquote{\_build/gettext}} directory.

\item {} 
Generate po files.

We’ll use the pot files generated in the above step.

\begin{sphinxVerbatim}[commandchars=\\\{\}]
\PYG{g+gp}{\PYGZdl{}} sphinx\PYGZhy{}intl update \PYGZhy{}p \PYGZus{}build/gettext \PYGZhy{}l de \PYGZhy{}l ja
\end{sphinxVerbatim}

Once completed, the generated po files will be placed in the below
directories:
\begin{itemize}
\item {} 
\sphinxcode{\sphinxupquote{./locale/de/LC\_MESSAGES/}}

\item {} 
\sphinxcode{\sphinxupquote{./locale/ja/LC\_MESSAGES/}}

\end{itemize}

\item {} 
Translate po files.

AS noted above, these are located in the \sphinxcode{\sphinxupquote{./locale/\textless{}lang\textgreater{}/LC\_MESSAGES}}
directory.  An example of one such file, from Sphinx, \sphinxcode{\sphinxupquote{builders.po}}, is
given below.

\begin{sphinxVerbatim}[commandchars=\\\{\}]
\PYG{c+c1}{\PYGZsh{} a5600c3d2e3d48fc8c261ea0284db79b}
\PYG{k+kd}{\PYGZsh{}: ../../builders.rst:4}
\PYG{n+nv}{msgid} \PYG{l+s}{\PYGZdq{}Available builders\PYGZdq{}}
\PYG{n+nv}{msgstr} \PYG{l+s}{\PYGZdq{}\PYGZlt{}FILL HERE BY TARGET LANGUAGE\PYGZgt{}\PYGZdq{}}
\end{sphinxVerbatim}

Another case, msgid is multi\sphinxhyphen{}line text and contains reStructuredText syntax:

\begin{sphinxVerbatim}[commandchars=\\\{\}]
\PYG{c+c1}{\PYGZsh{} 302558364e1d41c69b3277277e34b184}
\PYG{k+kd}{\PYGZsh{}: ../../builders.rst:9}
\PYG{n+nv}{msgid} \PYG{l+s}{\PYGZdq{}\PYGZdq{}}
\PYG{l+s}{\PYGZdq{}These are the built\PYGZhy{}in Sphinx builders. More builders can be added by \PYGZdq{}}
\PYG{l+s}{\PYGZdq{}:ref:`extensions \PYGZlt{}extensions\PYGZgt{}`.\PYGZdq{}}
\PYG{n+nv}{msgstr} \PYG{l+s}{\PYGZdq{}\PYGZdq{}}
\PYG{l+s}{\PYGZdq{}FILL HERE BY TARGET LANGUAGE FILL HERE BY TARGET LANGUAGE FILL HERE \PYGZdq{}}
\PYG{l+s}{\PYGZdq{}BY TARGET LANGUAGE :ref:`EXTENSIONS \PYGZlt{}extensions\PYGZgt{}` FILL HERE.\PYGZdq{}}
\end{sphinxVerbatim}

Please be careful not to break reST notation.  Most po\sphinxhyphen{}editors will help you
with that.

\item {} 
Build translated document.

You need a \sphinxcode{\sphinxupquote{language}} parameter in \sphinxcode{\sphinxupquote{conf.py}} or you may also
specify the parameter on the command line.

For for BSD/GNU make, run:

\begin{sphinxVerbatim}[commandchars=\\\{\}]
\PYG{g+gp}{\PYGZdl{}} make \PYGZhy{}e \PYG{n+nv}{SPHINXOPTS}\PYG{o}{=}\PYG{l+s+s2}{\PYGZdq{}\PYGZhy{}D language=\PYGZsq{}de\PYGZsq{}\PYGZdq{}} html
\end{sphinxVerbatim}

For Windows \sphinxstyleliteralstrong{\sphinxupquote{cmd.exe}}, run:

\begin{sphinxVerbatim}[commandchars=\\\{\}]
\PYG{g+gp}{\PYGZgt{}} \PYG{n+nb}{set} \PYG{n+nv}{SPHINXOPTS}\PYG{o}{=}\PYGZhy{}D \PYG{n+nv}{language}\PYG{o}{=}de
\PYG{g+gp}{\PYGZgt{}} .\PYG{l+s+se}{\PYGZbs{}m}ake.bat html
\end{sphinxVerbatim}

For PowerShell, run:

\begin{sphinxVerbatim}[commandchars=\\\{\}]
\PYG{g+gp}{\PYGZgt{}} Set\PYGZhy{}Item env:SPHINXOPTS \PYG{l+s+s2}{\PYGZdq{}\PYGZhy{}D language=de\PYGZdq{}}
\PYG{g+gp}{\PYGZgt{}} .\PYG{l+s+se}{\PYGZbs{}m}ake.bat html
\end{sphinxVerbatim}

\end{enumerate}

Congratulations! You got the translated documentation in the \sphinxcode{\sphinxupquote{\_build/html}}
directory.

\DUrole{versionmodified,added}{New in version 1.3: }\sphinxstyleliteralstrong{\sphinxupquote{sphinx\sphinxhyphen{}build}} that is invoked by make command will build po files
into mo files.

If you are using 1.2.x or earlier, please invoke \sphinxstyleliteralstrong{\sphinxupquote{sphinx\sphinxhyphen{}intl build}}
command before \sphinxstyleliteralstrong{\sphinxupquote{make}} command.


\subsubsection{Translating}
\label{\detokenize{usage/advanced/intl:translating}}

\subsubsection{Update your po files by new pot files}
\label{\detokenize{usage/advanced/intl:update-your-po-files-by-new-pot-files}}
If a document is updated, it is necessary to generate updated pot files and to
apply differences to translated po files.  In order to apply the updates from a
pot file to the po file, use the \sphinxstyleliteralstrong{\sphinxupquote{sphinx\sphinxhyphen{}intl update}} command.

\begin{sphinxVerbatim}[commandchars=\\\{\}]
\PYG{g+gp}{\PYGZdl{}} sphinx\PYGZhy{}intl update \PYGZhy{}p \PYGZus{}build/locale
\end{sphinxVerbatim}


\subsection{Using Transifex service for team translation}
\label{\detokenize{usage/advanced/intl:using-transifex-service-for-team-translation}}
\sphinxhref{https://www.transifex.com/}{Transifex}%
\begin{footnote}[6]\sphinxAtStartFootnote
\sphinxnolinkurl{https://www.transifex.com/}
%
\end{footnote} is one of several services that allow collaborative translation via a
web interface.  It has a nifty Python\sphinxhyphen{}based command line client that makes it
easy to fetch and push translations.
\begin{enumerate}
\sphinxsetlistlabels{\arabic}{enumi}{enumii}{}{.}%
\item {} 
Install \sphinxhref{https://pypi.org/project/transifex-client/}{transifex\sphinxhyphen{}client}%
\begin{footnote}[7]\sphinxAtStartFootnote
\sphinxnolinkurl{https://pypi.org/project/transifex-client/}
%
\end{footnote}.

You need \sphinxstyleliteralstrong{\sphinxupquote{tx}} command to upload resources (pot files).

\begin{sphinxVerbatim}[commandchars=\\\{\}]
\PYG{g+gp}{\PYGZdl{}} pip install transifex\PYGZhy{}client
\end{sphinxVerbatim}


\sphinxstrong{See also:}


\sphinxhref{https://docs.transifex.com/client/introduction/}{Transifex Client documentation}%
\begin{footnote}[8]\sphinxAtStartFootnote
\sphinxnolinkurl{https://docs.transifex.com/client/introduction/}
%
\end{footnote}



\item {} 
Create your \sphinxhref{https://www.transifex.com/}{transifex}%
\begin{footnote}[9]\sphinxAtStartFootnote
\sphinxnolinkurl{https://www.transifex.com/}
%
\end{footnote} account and create new project for your document.

Currently, transifex does not allow for a translation project to have more
than one version of the document, so you’d better include a version number in
your project name.

For example:
\begin{quote}\begin{description}
\item[{Project ID}] \leavevmode
\sphinxcode{\sphinxupquote{sphinx\sphinxhyphen{}document\sphinxhyphen{}test\_1\_0}}

\item[{Project URL}] \leavevmode
\sphinxcode{\sphinxupquote{https://www.transifex.com/projects/p/sphinx\sphinxhyphen{}document\sphinxhyphen{}test\_1\_0/}}

\end{description}\end{quote}

\item {} 
Create config files for \sphinxstyleliteralstrong{\sphinxupquote{tx}} command.

This process will create \sphinxcode{\sphinxupquote{.tx/config}} in the current directory, as well as
a \sphinxcode{\sphinxupquote{\textasciitilde{}/.transifexrc}} file that includes auth information.

\begin{sphinxVerbatim}[commandchars=\\\{\}]
\PYG{g+gp}{\PYGZdl{}} tx init
\PYG{g+go}{Creating .tx folder...}
\PYG{g+go}{Transifex instance [https://www.transifex.com]:}
\PYG{g+go}{...}
\PYG{g+go}{Please enter your transifex username: \PYGZlt{}transifex\PYGZhy{}username\PYGZgt{}}
\PYG{g+go}{Password: \PYGZlt{}transifex\PYGZhy{}password\PYGZgt{}}
\PYG{g+go}{...}
\PYG{g+go}{Done.}
\end{sphinxVerbatim}

\item {} 
Upload pot files to transifex service.

Register pot files to \sphinxcode{\sphinxupquote{.tx/config}} file:

\begin{sphinxVerbatim}[commandchars=\\\{\}]
\PYG{g+gp}{\PYGZdl{}} \PYG{n+nb}{cd} /your/document/root
\PYG{g+gp}{\PYGZdl{}} sphinx\PYGZhy{}intl update\PYGZhy{}txconfig\PYGZhy{}resources \PYGZhy{}\PYGZhy{}pot\PYGZhy{}dir \PYGZus{}build/locale \PYG{l+s+se}{\PYGZbs{}}
  \PYGZhy{}\PYGZhy{}transifex\PYGZhy{}project\PYGZhy{}name sphinx\PYGZhy{}document\PYGZhy{}test\PYGZus{}1\PYGZus{}0
\end{sphinxVerbatim}

and upload pot files:

\begin{sphinxVerbatim}[commandchars=\\\{\}]
\PYG{g+gp}{\PYGZdl{}} tx push \PYGZhy{}s
\PYG{g+go}{Pushing translations for resource sphinx\PYGZhy{}document\PYGZhy{}test\PYGZus{}1\PYGZus{}0.builders:}
\PYG{g+go}{Pushing source file (locale/pot/builders.pot)}
\PYG{g+go}{Resource does not exist.  Creating...}
\PYG{g+go}{...}
\PYG{g+go}{Done.}
\end{sphinxVerbatim}

\item {} 
Forward the translation on transifex.

\item {} 
Pull translated po files and make translated HTML.

Get translated catalogs and build mo files. For example, to build mo files
for German (de):

\begin{sphinxVerbatim}[commandchars=\\\{\}]
\PYG{g+gp}{\PYGZdl{}} \PYG{n+nb}{cd} /your/document/root
\PYG{g+gp}{\PYGZdl{}} tx pull \PYGZhy{}l de
\PYG{g+go}{Pulling translations for resource sphinx\PYGZhy{}document\PYGZhy{}test\PYGZus{}1\PYGZus{}0.builders (...)}
\PYG{g+go}{ \PYGZhy{}\PYGZgt{} de: locale/de/LC\PYGZus{}MESSAGES/builders.po}
\PYG{g+go}{...}
\PYG{g+go}{Done.}
\end{sphinxVerbatim}

Invoke \sphinxstyleliteralstrong{\sphinxupquote{make html}} (for BSD/GNU make):

\begin{sphinxVerbatim}[commandchars=\\\{\}]
\PYG{g+gp}{\PYGZdl{}} make \PYGZhy{}e \PYG{n+nv}{SPHINXOPTS}\PYG{o}{=}\PYG{l+s+s2}{\PYGZdq{}\PYGZhy{}D language=\PYGZsq{}de\PYGZsq{}\PYGZdq{}} html
\end{sphinxVerbatim}

\end{enumerate}

That’s all!

\begin{sphinxadmonition}{tip}{Tip:}
Translating locally and on Transifex

If you want to push all language’s po files, you can be done by using
\sphinxstyleliteralstrong{\sphinxupquote{tx push \sphinxhyphen{}t}} command.  Watch out! This operation overwrites
translations in transifex.

In other words, if you have updated each in the service and local po files,
it would take much time and effort to integrate them.
\end{sphinxadmonition}


\subsection{Contributing to Sphinx reference translation}
\label{\detokenize{usage/advanced/intl:contributing-to-sphinx-reference-translation}}
The recommended way for new contributors to translate Sphinx reference is to
join the translation team on Transifex.

There is \sphinxhref{https://www.transifex.com/sphinx-doc/sphinx-doc/}{sphinx translation page}%
\begin{footnote}[10]\sphinxAtStartFootnote
\sphinxnolinkurl{https://www.transifex.com/sphinx-doc/sphinx-doc/}
%
\end{footnote} for Sphinx (master) documentation.
\begin{enumerate}
\sphinxsetlistlabels{\arabic}{enumi}{enumii}{}{.}%
\item {} 
Login to \sphinxhref{https://www.transifex.com/}{transifex}%
\begin{footnote}[11]\sphinxAtStartFootnote
\sphinxnolinkurl{https://www.transifex.com/}
%
\end{footnote} service.

\item {} 
Go to \sphinxhref{https://www.transifex.com/sphinx-doc/sphinx-doc/}{sphinx translation page}%
\begin{footnote}[12]\sphinxAtStartFootnote
\sphinxnolinkurl{https://www.transifex.com/sphinx-doc/sphinx-doc/}
%
\end{footnote}.

\item {} 
Click \sphinxcode{\sphinxupquote{Request language}} and fill form.

\item {} 
Wait acceptance by transifex sphinx translation maintainers.

\item {} 
(After acceptance) Translate on transifex.

\end{enumerate}



\renewcommand{\indexname}{Index}

\IfFileExists{\jobname.ind}
             {\footnotesize\raggedright\printindex}
             {\begin{sphinxtheindex}\end{sphinxtheindex}}

\end{document}